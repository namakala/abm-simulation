% Options for packages loaded elsewhere
% Options for packages loaded elsewhere
\PassOptionsToPackage{unicode}{hyperref}
\PassOptionsToPackage{hyphens}{url}
\PassOptionsToPackage{dvipsnames,svgnames,x11names}{xcolor}
%
\documentclass[
  letterpaper,
  DIV=11,
  numbers=noendperiod]{scrartcl}
\usepackage{xcolor}
\usepackage{amsmath,amssymb}
\setcounter{secnumdepth}{-\maxdimen} % remove section numbering
\usepackage{iftex}
\ifPDFTeX
  \usepackage[T1]{fontenc}
  \usepackage[utf8]{inputenc}
  \usepackage{textcomp} % provide euro and other symbols
\else % if luatex or xetex
  \usepackage{unicode-math} % this also loads fontspec
  \defaultfontfeatures{Scale=MatchLowercase}
  \defaultfontfeatures[\rmfamily]{Ligatures=TeX,Scale=1}
\fi
\usepackage{lmodern}
\ifPDFTeX\else
  % xetex/luatex font selection
\fi
% Use upquote if available, for straight quotes in verbatim environments
\IfFileExists{upquote.sty}{\usepackage{upquote}}{}
\IfFileExists{microtype.sty}{% use microtype if available
  \usepackage[]{microtype}
  \UseMicrotypeSet[protrusion]{basicmath} % disable protrusion for tt fonts
}{}
\makeatletter
\@ifundefined{KOMAClassName}{% if non-KOMA class
  \IfFileExists{parskip.sty}{%
    \usepackage{parskip}
  }{% else
    \setlength{\parindent}{0pt}
    \setlength{\parskip}{6pt plus 2pt minus 1pt}}
}{% if KOMA class
  \KOMAoptions{parskip=half}}
\makeatother
% Make \paragraph and \subparagraph free-standing
\makeatletter
\ifx\paragraph\undefined\else
  \let\oldparagraph\paragraph
  \renewcommand{\paragraph}{
    \@ifstar
      \xxxParagraphStar
      \xxxParagraphNoStar
  }
  \newcommand{\xxxParagraphStar}[1]{\oldparagraph*{#1}\mbox{}}
  \newcommand{\xxxParagraphNoStar}[1]{\oldparagraph{#1}\mbox{}}
\fi
\ifx\subparagraph\undefined\else
  \let\oldsubparagraph\subparagraph
  \renewcommand{\subparagraph}{
    \@ifstar
      \xxxSubParagraphStar
      \xxxSubParagraphNoStar
  }
  \newcommand{\xxxSubParagraphStar}[1]{\oldsubparagraph*{#1}\mbox{}}
  \newcommand{\xxxSubParagraphNoStar}[1]{\oldsubparagraph{#1}\mbox{}}
\fi
\makeatother


\usepackage{longtable,booktabs,array}
\usepackage{calc} % for calculating minipage widths
% Correct order of tables after \paragraph or \subparagraph
\usepackage{etoolbox}
\makeatletter
\patchcmd\longtable{\par}{\if@noskipsec\mbox{}\fi\par}{}{}
\makeatother
% Allow footnotes in longtable head/foot
\IfFileExists{footnotehyper.sty}{\usepackage{footnotehyper}}{\usepackage{footnote}}
\makesavenoteenv{longtable}
\usepackage{graphicx}
\makeatletter
\newsavebox\pandoc@box
\newcommand*\pandocbounded[1]{% scales image to fit in text height/width
  \sbox\pandoc@box{#1}%
  \Gscale@div\@tempa{\textheight}{\dimexpr\ht\pandoc@box+\dp\pandoc@box\relax}%
  \Gscale@div\@tempb{\linewidth}{\wd\pandoc@box}%
  \ifdim\@tempb\p@<\@tempa\p@\let\@tempa\@tempb\fi% select the smaller of both
  \ifdim\@tempa\p@<\p@\scalebox{\@tempa}{\usebox\pandoc@box}%
  \else\usebox{\pandoc@box}%
  \fi%
}
% Set default figure placement to htbp
\def\fps@figure{htbp}
\makeatother


% definitions for citeproc citations
\NewDocumentCommand\citeproctext{}{}
\NewDocumentCommand\citeproc{mm}{%
  \begingroup\def\citeproctext{#2}\cite{#1}\endgroup}
\makeatletter
 % allow citations to break across lines
 \let\@cite@ofmt\@firstofone
 % avoid brackets around text for \cite:
 \def\@biblabel#1{}
 \def\@cite#1#2{{#1\if@tempswa , #2\fi}}
\makeatother
\newlength{\cslhangindent}
\setlength{\cslhangindent}{1.5em}
\newlength{\csllabelwidth}
\setlength{\csllabelwidth}{3em}
\newenvironment{CSLReferences}[2] % #1 hanging-indent, #2 entry-spacing
 {\begin{list}{}{%
  \setlength{\itemindent}{0pt}
  \setlength{\leftmargin}{0pt}
  \setlength{\parsep}{0pt}
  % turn on hanging indent if param 1 is 1
  \ifodd #1
   \setlength{\leftmargin}{\cslhangindent}
   \setlength{\itemindent}{-1\cslhangindent}
  \fi
  % set entry spacing
  \setlength{\itemsep}{#2\baselineskip}}}
 {\end{list}}
\usepackage{calc}
\newcommand{\CSLBlock}[1]{\hfill\break\parbox[t]{\linewidth}{\strut\ignorespaces#1\strut}}
\newcommand{\CSLLeftMargin}[1]{\parbox[t]{\csllabelwidth}{\strut#1\strut}}
\newcommand{\CSLRightInline}[1]{\parbox[t]{\linewidth - \csllabelwidth}{\strut#1\strut}}
\newcommand{\CSLIndent}[1]{\hspace{\cslhangindent}#1}



\setlength{\emergencystretch}{3em} % prevent overfull lines

\providecommand{\tightlist}{%
  \setlength{\itemsep}{0pt}\setlength{\parskip}{0pt}}



 


\usepackage{longtable}
\usepackage{amsmath}
\usepackage{amssymb}
\usepackage{multicol}
\usepackage{float}
\usepackage{typearea}
\floatplacement{figure}{htbp}
\floatplacement{table}{htbp}
\AtBeginDocument{%
  \storeareas\normalpapersize
}
\BeforeRestoreareas{\cleardoublepage}
\newcommand*\uselandscape{%
  \cleardoublepage
  \KOMAoptions{paper=landscape}%
  \recalctypearea
  \areaset{1.2\textwidth}{1.2\textheight}%
}
\KOMAoption{captions}{tableheading}
\makeatletter
\@ifpackageloaded{caption}{}{\usepackage{caption}}
\AtBeginDocument{%
\ifdefined\contentsname
  \renewcommand*\contentsname{Table of contents}
\else
  \newcommand\contentsname{Table of contents}
\fi
\ifdefined\listfigurename
  \renewcommand*\listfigurename{List of Figures}
\else
  \newcommand\listfigurename{List of Figures}
\fi
\ifdefined\listtablename
  \renewcommand*\listtablename{List of Tables}
\else
  \newcommand\listtablename{List of Tables}
\fi
\ifdefined\figurename
  \renewcommand*\figurename{Figure}
\else
  \newcommand\figurename{Figure}
\fi
\ifdefined\tablename
  \renewcommand*\tablename{Table}
\else
  \newcommand\tablename{Table}
\fi
}
\@ifpackageloaded{float}{}{\usepackage{float}}
\floatstyle{ruled}
\@ifundefined{c@chapter}{\newfloat{codelisting}{h}{lop}}{\newfloat{codelisting}{h}{lop}[chapter]}
\floatname{codelisting}{Listing}
\newcommand*\listoflistings{\listof{codelisting}{List of Listings}}
\makeatother
\makeatletter
\makeatother
\makeatletter
\@ifpackageloaded{caption}{}{\usepackage{caption}}
\@ifpackageloaded{subcaption}{}{\usepackage{subcaption}}
\makeatother
\usepackage{bookmark}
\IfFileExists{xurl.sty}{\usepackage{xurl}}{} % add URL line breaks if available
\urlstyle{same}
\hypersetup{
  pdftitle={An agent-based simulation of psychological resilience},
  pdfauthor={Aly Lamuri},
  colorlinks=true,
  linkcolor={blue},
  filecolor={Maroon},
  citecolor={Blue},
  urlcolor={Blue},
  pdfcreator={LaTeX via pandoc}}


\title{An agent-based simulation of psychological resilience}
\author{Aly Lamuri}
\date{}
\begin{document}
\maketitle


\section{Introduction}\label{introduction}

Anxiety and depression are among the most prevalent psychological
disorders in the general population, generating substantial
socioeconomic consequences (GBD Collaborators 2022; Arias, Saxena, and
Verguet 2022). These conditions impose considerable direct and indirect
costs, while curative and rehabilitative approaches often demonstrate
limited effectiveness (Mekonen et al. 2021). Strengthening psychological
resilience offers a preventive pathway to reduce vulnerability to such
disorders (Troy et al. 2023). Public health policy should therefore
prioritize preventive strategies that foster resilience. However,
designing cost-effective preventive policies requires rigorous
assessment methods. Agent-based modeling (ABM) provides a promising
approach to simulate psychological resilience and generate evidence to
guide interventions aimed at enhancing resilience at the population
level.

\subsection{Previous ABM Publications}\label{previous-abm-publications}

The following five articles were selected for their relevance to the
development of ABM simulations addressing mental health, social
interaction, or their intersection. Each study contributed unique
methodological insights that informed the current work. While some
articles demonstrated only minimal overlap with our research, they were
included because they resolved specific modeling challenges pertinent to
our framework. Collectively, these studies provided a foundation for
constructing comprehensive and realistic ABM simulations in the context
of mental health and social dynamics.

Romanyukha et al. (2023) implemented an ABM to evaluate the mental
health consequences of policy measures during the COVID-19 pandemic. The
model integrated epidemic dynamics with the development of mental
disorders in a large urban population. Using Quality-Adjusted Life Years
(QALY) as an outcome measure, the study considered major depressive
disorder, anxiety disorder, COVID-19 cases (lethal and non-lethal), and
immunization. The results showed that mental disorders accounted for a
substantial share of pandemic-related health losses, and under strong
lockdown conditions, the lowest QALY loss occurred when approximately
70\% of the population was isolated. These findings demonstrate that ABM
can quantify the impact of policy interventions on mental health
outcomes, supporting its use for evaluating preventive strategies that
enhance psychological resilience.

Murase et al. (2021) introduced a generalized weighted social network
(GWSN) model to describe the formation of social ties and networks. The
GWSN model integrates mechanisms previously examined in isolation,
including triadic closure (the tendency for friends of friends to become
friends), homophilic interactions, and link termination processes. Due
to the model's complexity and numerous input parameters, the authors
combined extensive simulations with deep neural networks for regression
and global sensitivity analysis, enabling prediction of network
properties and identification of key determinants of network dynamics.
By providing a robust framework for modeling realistic social
interactions, this approach is directly relevant for simulating how
social connectivity influences resilience and the spread of mental
disorders in ABM.

Andarlia and Gunawan (2021) examined the contagion effect of
psychological disorders using ABM to simulate the spread of depression
in a population. The model accounted for the influence of close social
relationships, which could trigger mild, moderate, or severe depressive
episodes, and modeled recovery through therapy with transition dynamics
and recovery rates differentiated by gender. Simulation results
indicated that higher contact rates with depressed individuals increased
episode severity, whereas greater therapy uptake reduced depression
prevalence. This study highlights how ABM can capture both the
propagation and mitigation of mental disorders through social
mechanisms, a key component for modeling interventions aimed at
strengthening psychological resilience.

López and Giovanini (2024) introduced a methodology embedding an ABM
within adaptive temporal networks to simulate daily interactions and
examine the interplay between infectious disease spread and individual
behavior. The model integrates individual behavior, social dynamics, and
epidemiological factors, validated using real-world influenza outbreak
data, and can simulate complex social phenomena such as social
awareness. By incorporating self-organized system logic, agents respond
dynamically to external stimuli, perceptions, and health states.
Combined with social network models and mental health contagion
dynamics, this approach demonstrates that ABM can simulate emergent
phenomena arising from the interaction of social behavior and health,
providing a powerful tool for testing policies that influence population
resilience.

Benny et al. (2022) employed ABM to evaluate the impact of government
policies on mental health, focusing on depression in expectant mothers.
The model incorporated parameters from the ``All Our Families'' cohort
dataset and literature, including household income, age, education, and
environmental factors. The study hypothesized that income-supportive
programs would reduce maternal depression, and simulations confirmed
that progressive income policies significantly lowered prevalence.
Additionally, modeling mothers' social networks showed that expanding
social connections further reduced depression risk. These findings
illustrate that ABM can assess the effectiveness of policy interventions
on mental health outcomes, underscoring its applicability for designing
strategies to enhance resilience and reduce the prevalence of
psychological disorders.

\subsection{Current Research}\label{current-research}

This study aims to simulate stress perception using an ABM approach.
Stress perception is operationalized as a generalization of the
Perceived Stress Scale-10, following the guidance provided in its
interpretation manual (Cohen 1988). Modeling stress perception is
essential for capturing the activation of psychological resilience
(Parsons, Kruijt, and Fox 2016). Under equivalent stress exposure,
agents with higher resilience are less likely to develop psychological
disorders, whereas less resilient agents are more vulnerable. The
simulation incorporates multiple psychological resources, including
social support, family support, and psychological capital, with the
latter encompassing personality traits such as openness,
conscientiousness, and extraversion (Hobfoll 1989; Klaver et al. 2021;
Zager Kocjan, Kavčič, and Avsec 2021).

\section{Methods}\label{methods}

\subsection{Affect}\label{affect}

Affect represents an individual's current emotional state, ranging from
very negative to very positive. The model examines how various factors
influence these emotional states, including social connections, stress
events, and natural tendencies to return to an individual's baseline
emotional level.

\subsubsection{Social Influence on
Affect}\label{social-influence-on-affect}

The affect of agent \(i\) at time \(t+1\) is influenced by the affect of
neighboring agents \(j \in N_i\):

\[\Delta a_{i,t+1}^{social} = \alpha \sum_{j \in N_i} (a_{j,t} - a_{i,t})\]

where \(\alpha\) represents the peer influence rate and \(N_i\) is the
set of neighboring agents.

\subsubsection{Stress Event Impact}\label{stress-event-impact}

Stress events modify affect based on challenge-hindrance appraisal and
coping outcomes:

\[\Delta a_{i,t+1}^{stress} = \beta \cdot (c_i \cdot s_i - h_i \cdot f_i)\]

where \(\beta\) is the event appraisal rate, \(c_i\) and \(h_i\)
represent challenge and hindrance components, \(s_i\) is coping success
(1 for success, 0 for failure), and \(f_i\) is coping failure (1 for
failure, 0 for success).

\subsubsection{Homeostatic Regulation}\label{homeostatic-regulation}

Affect naturally returns to baseline through homeostatic processes:

\[a_{i,t+1} = a_{i,t} + \gamma (a_{i,0} - a_{i,t}) + \Delta a_{i,t+1}^{social} + \Delta a_{i,t+1}^{stress}\]

where \(\gamma\) is the homeostatic rate and \(a_{i,0}\) is the baseline
affect level.

\subsubsection{Resilience-Affect
Integration}\label{resilience-affect-integration}

Positive affect enhances resource regeneration, while high resilience
provides emotional buffering:

\[r_{i,t+1} = r_{i,t} + \delta \cdot \max(0, a_{i,t}) \cdot R_{max} - \epsilon \cdot \max(0, -a_{i,t})\]

where \(r_i\) represents resources, \(\delta\) is the positive affect
regeneration bonus, \(\epsilon\) is the negative affect regeneration
penalty, and \(R_{max}\) is maximum resource capacity.

\subsection{Stress Perception}\label{stress-perception}

Stress perception follows a comprehensive pipeline that transforms life
events into psychological stress responses using the challenge-hindrance
framework.

\subsubsection{Challenge-Hindrance
Appraisal}\label{challenge-hindrance-appraisal}

Each stress event \(e_i\) is characterized by controllability
\(c_i \in [0,1]\) and overload \(o_i \in [0,1]\):

\[z_i = \omega_c c_i - \omega_o o_i + b\]

\[challenge_i = \sigma(\gamma z_i) = \frac{1}{1 + e^{-\gamma z_i}}\]

\[hindrance_i = 1 - challenge_i\]

where \(\omega_c\) and \(\omega_o\) are weight parameters, \(b\) is a
bias term, and \(\gamma\) controls sigmoid steepness.

\subsubsection{PSS-10 Bifactor Model}\label{pss-10-bifactor-model}

The PSS-10 integrates controllability and overload dimensions using a
bifactor model:

\[PSS_{10_{total}} = \sum_{k=1}^{10} item_k\]

\[item_k = \mu_k + \lambda_k^c \cdot C + \lambda_k^o \cdot O + \epsilon_k\]

where \(C \sim N(0,1)\) and \(O \sim N(0,1)\) are latent factors with
correlation \(\rho\), \(\lambda_k^c\) and \(\lambda_k^o\) are factor
loadings, \(\mu_k\) are item means, and \(\epsilon_k\) are residuals.

\subsubsection{Stress Threshold
Evaluation}\label{stress-threshold-evaluation}

An agent becomes stressed when appraised stress load exceeds their
dynamic threshold:

\[L_i = s_i \cdot (1 + \delta \cdot (hindrance_i - challenge_i))\]

\[T_{i,effective} = T_{base} + \lambda_C \cdot challenge_i - \lambda_H \cdot hindrance_i\]

\[stressed_i = \begin{cases} 1 & \text{if } L_i > T_{i,effective} \\ 0 & \text{otherwise} \end{cases}\]

where \(s_i\) is event magnitude, \(\delta\) controls
challenge-hindrance polarity, \(\lambda_C\) and \(\lambda_H\) are
threshold scaling parameters, and \(T_{base}\) is the baseline
threshold.

\subsection{Resilience}\label{resilience}

Resilience represents an individual's capacity to adapt and recover from
stress events, with state variables including current resilience level,
baseline resilience, consecutive hindrances, and stress breach count.

\subsubsection{Resilience Changes from Coping
Outcomes}\label{resilience-changes-from-coping-outcomes}

Resilience updates based on challenge-hindrance appraisal and coping
success:

\[\Delta r_i = \begin{cases}
\rho_{success} \cdot challenge_i & \text{if coping successful} \\
\rho_{minor} \cdot hindrance_i & \text{if coping successful} \\
-\rho_{failure} \cdot hindrance_i & \text{if coping failed} \\
-\rho_{minor} \cdot challenge_i & \text{if coping failed}
\end{cases}\]

where \(\rho_{success}\), \(\rho_{minor}\), \(\rho_{failure}\) are rate
parameters for different outcomes.

\subsubsection{Social Support Boost}\label{social-support-boost}

Social support provides resilience enhancement:

\[r_{i,t+1} = r_{i,t} + \sigma_{support} \cdot \sum_{j \in supporters} q_j\]

where \(\sigma_{support}\) is the social support rate and \(q_j\) is the
quality of support from agent \(j\).

\subsubsection{Overload Effects}\label{overload-effects}

Cumulative hindrance events create overload when exceeding threshold:

\[overload_i = \begin{cases} 1 & \text{if } h_i^{consecutive} > \theta_{overload} \\ 0 & \text{otherwise} \end{cases}\]

\[\Delta r_{i,overload} = -\nu \cdot overload_i\]

where \(\theta_{overload}\) is the overload threshold and \(\nu\) is the
overload penalty rate.

\subsubsection{Homeostatic Adjustment}\label{homeostatic-adjustment}

Resilience returns to baseline over time:

\[r_{i,t+1} = r_{i,t} + \eta (r_{i,0} - r_{i,t}) + \Delta r_i + \Delta r_{i,overload}\]

where \(\eta\) is the homeostatic rate and \(r_{i,0}\) is baseline
resilience.

\subsubsection{Coping Success
Determination}\label{coping-success-determination}

Coping probability integrates challenge-hindrance effects and social
influence:

\[p_{coping_i} = p_{base} + \phi_C \cdot challenge_i - \phi_H \cdot hindrance_i + \psi \cdot \bar{a}_{N_i}\]

where \(p_{base}\) is base coping probability, \(\phi_C\) and \(\phi_H\)
are challenge and hindrance modifiers, \(\psi\) is social influence
strength, and \(\bar{a}_{N_i}\) is average neighbor affect.

\subsection{Agent Interaction}\label{agent-interaction}

Agent interactions form the core social dynamics through a grid-based
social network reflecting real-world spatial relationships.

\subsubsection{Network Structure}\label{network-structure}

Agents are positioned on a 2D grid with connections to Moore neighbors:

\[N_i = \{j \mid \max(|x_i - x_j|, |y_i - y_j|) \leq 1, j \neq i\}\]

where \((x_i, y_i)\) are grid coordinates of agent \(i\).

\subsubsection{Social Influence
Dynamics}\label{social-influence-dynamics}

Social interactions create bidirectional affect changes with asymmetric
effects:

\[\Delta a_{i,t+1} = \mu \sum_{j \in N_i} (a_{j,t} - a_{i,t})\]

\[\Delta a_{j,t+1} = \mu \sum_{i \in N_j} (a_{i,t} - a_{j,t})\]

where \(\mu\) is the interaction influence rate.

\subsubsection{Network Adaptation}\label{network-adaptation}

Stress triggers network rewiring based on homophily and support
effectiveness:

\[p_{rewire_i} = \begin{cases}
\pi & \text{if } stress_i^{breaches} > \theta_{adapt} \\ 0 & \text{otherwise} \end{cases}\]

\subsubsection{Homophily-Based
Connection}\label{homophily-based-connection}

Connection probability considers stress and affect similarity:

\[p_{connect_{ij}} = \frac{1}{1 + e^{-\kappa (similarity_{ij} - 0.5)}}\]

\[similarity_{ij} = 1 - \frac{|stress_i - stress_j| + |a_i - a_j|}{2}\]

where \(\kappa\) controls homophily strength.

\subsubsection{Social Support Dynamics}\label{social-support-dynamics}

Support provision depends on neighbor affect and resilience:

\[support_{ij} = \begin{cases}
q_j \cdot r_j & \text{if } a_j > \theta_{support} \\ 0 & \text{otherwise} \end{cases}\]

where \(q_j\) is support quality and \(\theta_{support}\) is the support
threshold.

\subsection{Psychological Resource
Management}\label{psychological-resource-management}

Resource management implements conservation of resources theory, where
individuals have limited psychological and physical resources to
allocate across protective strategies.

\subsubsection{Resource Regeneration}\label{resource-regeneration}

Resources regenerate toward maximum capacity with affect modulation:

\[R_{i,t+1} = R_{i,t} + \rho (R_{max} - R_{i,t}) + \delta \cdot \max(0, a_{i,t}) \cdot R_{max}\]

where \(\rho\) is the base regeneration rate, \(R_{max}\) is maximum
resource capacity, and \(\delta\) is the positive affect bonus.

\subsubsection{Resource Consumption}\label{resource-consumption}

Successful coping consumes resources proportional to event hindrance:

\[R_{i,t+1} = R_{i,t} - \kappa \cdot hindrance_i \cdot s_i\]

where \(\kappa\) is the resource cost parameter and \(s_i\) indicates
coping success.

\subsubsection{Softmax Resource
Allocation}\label{softmax-resource-allocation}

Resources are allocated across protective factors using softmax
decision-making:

\[allocation_{i,k} = \frac{e^{u_{i,k}/\tau}}{\sum_{m=1}^4 e^{u_{i,m}/\tau}} \cdot R_{i,t}\]

\[u_{i,k} = \alpha_k \cdot (1 - p_{i,k}) + \beta \cdot (1 - resilience_i)\]

where \(\tau\) is the softmax temperature, \(u_{i,k}\) is the utility of
protective factor \(k\), \(\alpha_k\) is the efficacy of factor \(k\),
\(p_{i,k}\) is the current protection level, and \(\beta\) controls
need-based allocation.

\subsubsection{Protective Factor
Dynamics}\label{protective-factor-dynamics}

Protective factors improve based on resource investment:

\[\Delta p_{i,k} = \gamma_k \cdot allocation_{i,k} \cdot (1 - p_{i,k})\]

where \(\gamma_k\) is the improvement rate for protective factor \(k\).

\textbf{Gain/Loss Spirals:} The model captures resource spirals through
coupled dynamics:

\textbf{Gain Spiral:}
\(R \uparrow \rightarrow allocation \uparrow \rightarrow p \uparrow \rightarrow coping \uparrow \rightarrow a \uparrow \rightarrow regeneration \uparrow\)

\textbf{Loss Spiral:}
\(R \downarrow \rightarrow allocation \downarrow \rightarrow p \downarrow \rightarrow coping \downarrow \rightarrow a \downarrow \rightarrow regeneration \downarrow\)

\subsection{Parameters}\label{parameters}

\begin{longtable}[]{@{}
  >{\raggedright\arraybackslash}p{(\linewidth - 6\tabcolsep) * \real{0.3390}}
  >{\raggedright\arraybackslash}p{(\linewidth - 6\tabcolsep) * \real{0.1864}}
  >{\raggedright\arraybackslash}p{(\linewidth - 6\tabcolsep) * \real{0.2542}}
  >{\raggedright\arraybackslash}p{(\linewidth - 6\tabcolsep) * \real{0.2203}}@{}}
\toprule\noalign{}
\begin{minipage}[b]{\linewidth}\raggedright
Parameter Category
\end{minipage} & \begin{minipage}[b]{\linewidth}\raggedright
Parameter
\end{minipage} & \begin{minipage}[b]{\linewidth}\raggedright
Default Value
\end{minipage} & \begin{minipage}[b]{\linewidth}\raggedright
Description
\end{minipage} \\
\midrule\noalign{}
\endhead
\bottomrule\noalign{}
\endlastfoot
\textbf{Simulation} & \texttt{SIMULATION\_NUM\_AGENTS} & 20 & Number of
agents in the simulation population \\
& \texttt{SIMULATION\_MAX\_DAYS} & 100 & Maximum simulation duration in
days \\
& \texttt{SIMULATION\_SEED} & 42 & Random seed for reproducible
results \\
\textbf{Network} & \texttt{NETWORK\_WATTS\_K} & 4 & Watts-Strogatz
network: neighbors per node \\
& \texttt{NETWORK\_WATTS\_P} & 0.1 & Watts-Strogatz network: rewiring
probability \\
& \texttt{NETWORK\_ADAPTATION\_THRESHOLD} & 3 & Threshold for network
adaptation when stress threshold breached \\
\textbf{Agent State} & \texttt{AGENT\_INITIAL\_RESILIENCE} & 0.5 &
Starting resilience level for all agents \\
& \texttt{AGENT\_INITIAL\_AFFECT} & 0.0 & Initial emotional state
(-1=negative, 1=positive) \\
& \texttt{AGENT\_INITIAL\_RESOURCES} & 0.6 & Starting psychological
resources \\
& \texttt{AGENT\_STRESS\_PROBABILITY} & 0.5 & Daily probability of
experiencing stress events \\
\textbf{Stress Events} & \texttt{STRESS\_CONTROLLABILITY\_MEAN} & 0.5 &
Mean controllability of stress events \\
& \texttt{STRESS\_OVERLOAD\_MEAN} & 0.5 & Mean overload intensity of
stress events \\
& \texttt{STRESS\_BETA\_ALPHA} & 2.0 & Beta distribution shape parameter
for controllability \\
& \texttt{STRESS\_BETA\_BETA} & 2.0 & Beta distribution shape parameter
for overload \\
\textbf{PSS-10} & \texttt{PSS10\_THRESHOLD} & 27 & Threshold score for
determining stressed state \\
& \texttt{PSS10\_BIFACTOR\_COR} & 0.3 & Correlation between
controllability and overload dimensions \\
\textbf{Appraisal} & \texttt{APPRAISAL\_OMEGA\_C} & 1.0 & Weight for
controllability in challenge/hindrance appraisal \\
& \texttt{APPRAISAL\_OMEGA\_O} & 1.0 & Weight for overload in
appraisal \\
& \texttt{APPRAISAL\_GAMMA} & 6.0 & Sigmoid steepness for
challenge/hindrance classification \\
\textbf{Thresholds} & \texttt{THRESHOLD\_BASE\_THRESHOLD} & 0.5 & Base
stress threshold for becoming stressed \\
& \texttt{THRESHOLD\_CHALLENGE\_SCALE} & 0.15 & Challenge threshold
modifier \\
& \texttt{THRESHOLD\_HINDRANCE\_SCALE} & 0.25 & Hindrance threshold
modifier \\
\textbf{Coping} & \texttt{COPING\_BASE\_PROBABILITY} & 0.5 & Base coping
probability before situational modifiers \\
& \texttt{COPING\_CHALLENGE\_BONUS} & 0.2 & Bonus to coping success for
challenge events \\
& \texttt{COPING\_HINDRANCE\_PENALTY} & 0.3 & Penalty to coping success
for hindrance events \\
\textbf{Interactions} & \texttt{INTERACTION\_INFLUENCE\_RATE} & 0.05 &
Base rate of affect influence between agents \\
& \texttt{INTERACTION\_MAX\_NEIGHBORS} & 10 & Maximum neighbors
considered for interactions \\
\textbf{Affect} & \texttt{AFFECT\_PEER\_INFLUENCE\_RATE} & 0.1 &
Strength of peer influence on affect \\
& \texttt{AFFECT\_EVENT\_APPRAISAL\_RATE} & 0.15 & How events affect
baseline affect through appraisal \\
& \texttt{AFFECT\_HOMEOSTATIC\_RATE} & 0.5 & Tendency for affect to
return to baseline \\
\textbf{Resilience} & \texttt{RESILIENCE\_COPING\_SUCCESS\_RATE} & 0.1 &
Resilience change from successful coping \\
& \texttt{RESILIENCE\_SOCIAL\_SUPPORT\_RATE} & 0.08 & Resilience boost
from social support \\
& \texttt{RESILIENCE\_OVERLOAD\_THRESHOLD} & 3 & Minimum consecutive
hindrances for overload effect \\
\textbf{Resources} & \texttt{PROTECTIVE\_SOCIAL\_SUPPORT} & 0.5 &
Efficacy of social support in reducing distress \\
& \texttt{PROTECTIVE\_FAMILY\_SUPPORT} & 0.5 & Efficacy of family
support \\
& \texttt{PROTECTIVE\_FORMAL\_INTERVENTION} & 0.5 & Efficacy of
professional interventions \\
& \texttt{RESOURCE\_BASE\_REGENERATION} & 0.05 & Daily resource
regeneration rate \\
\end{longtable}

\subsection{Model Integration}\label{model-integration}

The agent-based model integrates six core mechanism groups that work
together to simulate realistic mental health dynamics through a
comprehensive state transition framework.

\subsubsection{Agent State Vector}\label{agent-state-vector}

Each agent \(i\) maintains a complete state vector updated daily:

\[\mathbf{s}_{i,t+1} = f(\mathbf{s}_{i,t}, \mathbf{e}_{i,t}, \mathbf{n}_{i,t}, \mathbf{\theta})\]

where
\(\mathbf{s}_i = [a_i, r_i, stress_i, \mathbf{p}_i, R_i, pss10_i]\)
represents affect, resilience, stress level, protective factors,
resources, and PSS-10 scores; \(\mathbf{e}_i\) are external events;
\(\mathbf{n}_i\) are network influences; and \(\mathbf{\theta}\) are
model parameters.

\subsubsection{Daily Step Integration}\label{daily-step-integration}

Each simulation day follows a structured sequence:

\begin{enumerate}
\def\labelenumi{\arabic{enumi}.}
\tightlist
\item
  \textbf{Initialization:}
  \(\mathbf{s}_{i,t}^{init} = initialize(\mathbf{s}_{i,t-1})\)
\item
  \textbf{Event Processing:}
  \(\mathbf{e}_{i,t} \sim P(\mathbf{e}|\mathbf{s}_{i,t}^{init}, \mathbf{\theta})\)
\item
  \textbf{Social Interaction:}
  \(\Delta \mathbf{s}_{i,t}^{social} = g(\mathbf{s}_{N_i,t}, \mathbf{\theta}_{social})\)
\item
  \textbf{Stress Appraisal:}
  \(stress_{i,t} = appraise(\mathbf{e}_{i,t}, \mathbf{\theta}_{stress})\)
\item
  \textbf{Coping Determination:}
  \(coping_i = h(stress_{i,t}, r_{i,t}, \mathbf{s}_{N_i,t})\)
\item
  \textbf{State Updates:}
  \(\mathbf{s}_{i,t+1} = update(\mathbf{s}_{i,t}, \Delta \mathbf{s}_{i,t}^{social}, stress_{i,t}, coping_i)\)
\item
  \textbf{Homeostasis:}
  \(\mathbf{s}_{i,t+1} = homeostasis(\mathbf{s}_{i,t+1}, \mathbf{s}_{i,0})\)
\end{enumerate}

\subsubsection{Feedback Loops}\label{feedback-loops}

\textbf{Positive Reinforcement Loop:}
\[r_{t+1} = r_t + \phi_{success} \cdot challenge \cdot coping_{success}\]
\[a_{t+1} = a_t + \beta \cdot coping_{success} \cdot challenge\]
\[R_{t+1} = R_t + \delta \cdot \max(0, a_{t+1})\]

\textbf{Negative Degradation Loop:}
\[r_{t+1} = r_t - \phi_{failure} \cdot hindrance \cdot coping_{failure}\]
\[a_{t+1} = a_t - \beta \cdot coping_{failure} \cdot hindrance\]
\[R_{t+1} = R_t - \epsilon \cdot \max(0, -a_{t+1})\]

\textbf{Social Amplification Loop:}
\[\Delta a_i = \mu \sum_{j \in N_i} (a_j - a_i)\]
\[p_{connect_{ij}} = \frac{1}{1 + e^{-\kappa |a_i - a_j|}}\]

\textbf{Homeostatic Equilibria:} The system tends toward equilibrium
points where all mechanisms balance:

\[\frac{d\mathbf{s}}{dt} = 0 \implies \mathbf{s}^* = equilibrium(\mathbf{\theta})\]

where \(\mathbf{s}^*\) represents stable mental health states depending
on parameter configurations and social network structure.

\section*{References}\label{references}
\addcontentsline{toc}{section}{References}

\multicols{2}
\footnotesize

\phantomsection\label{refs}
\begin{CSLReferences}{1}{0}
\bibitem[\citeproctext]{ref-Andarlia2021}
Andarlia, H T, and I Gunawan. 2021. {``An Agent-Based Model of Contagion
Effects in Affected Depression and Its Recovery Process.''}
\emph{Journal of Physics: Conference Series} 1751 (1): 012007.
\url{https://doi.org/10.1088/1742-6596/1751/1/012007}.

\bibitem[\citeproctext]{ref-arias2022quantifying}
Arias, Daniel, Shekhar Saxena, and Stéphane Verguet. 2022.
{``Quantifying the Global Burden of Mental Disorders and Their Economic
Value.''} \emph{EClinicalMedicine} 54.

\bibitem[\citeproctext]{ref-Benny2022}
Benny, Claire, Shelby Yamamoto, Sheila McDonald, Radha Chari, and Roman
Pabayo. 2022. {``Modelling Maternal Depression: An Agent-Based Model to
Examine the Complex Relationship Between Relative Income and
Depression.''} \emph{International Journal of Environmental Research and
Public Health} 19 (7): 4208.
\url{https://doi.org/10.3390/ijerph19074208}.

\bibitem[\citeproctext]{ref-cohen1988perceived}
Cohen, Sheldon. 1988. {``Perceived Stress in a Probability Sample of the
United States.''}

\bibitem[\citeproctext]{ref-gbd2022global}
GBD Collaborators. 2022. {``Global, Regional, and National Burden of 12
Mental Disorders in 204 Countries and Territories, 1990--2019: A
Systematic Analysis for the Global Burden of Disease Study 2019.''}
\emph{The Lancet Psychiatry} 9 (2): 137--50.

\bibitem[\citeproctext]{ref-Hobfoll1989}
Hobfoll, Stevan E. 1989. {``Conservation of Resources: A New Attempt at
Conceptualizing Stress.''} \emph{American Psychologist} 44 (3): 513--24.
\url{https://doi.org/10.1037/0003-066x.44.3.513}.

\bibitem[\citeproctext]{ref-klaver2021exposure}
Klaver, M, BJ Van den Hoofdakker, H Wouters, G De Kuijper, PJ Hoekstra,
and A De Bildt. 2021. {``Exposure to Challenging Behaviours and Burnout
Symptoms Among Care Staff: The Role of Psychological Resources.''}
\emph{Journal of Intellectual Disability Research} 65 (2): 173--85.

\bibitem[\citeproctext]{ref-Lopez2024}
López, Leonardo, and Leonardo Giovanini. 2024. {``Adaptive Dynamic
Social Networks Using an Agent-Based Model to Study the Role of Social
Awareness in Infectious Disease Spread,''} July.
\url{https://doi.org/10.1101/2024.07.16.24310475}.

\bibitem[\citeproctext]{ref-mekonen2021estimating}
Mekonen, Tesfa, Gary CK Chan, Jason P Connor, Leanne Hides, and Janni
Leung. 2021. {``Estimating the Global Treatment Rates for Depression: A
Systematic Review and Meta-Analysis.''} \emph{Journal of Affective
Disorders} 295: 1234--42.

\bibitem[\citeproctext]{ref-Murase2021}
Murase, Yohsuke, Hang-Hyun Jo, János Török, János Kertész, and Kimmo
Kaski. 2021. {``Deep Learning Exploration of Agent-Based Social Network
Model Parameters.''} \emph{Frontiers in Big Data} 4 (September).
\url{https://doi.org/10.3389/fdata.2021.739081}.

\bibitem[\citeproctext]{ref-parsons2016cognitive}
Parsons, Sam, Anne-Wil Kruijt, and Elaine Fox. 2016. {``A Cognitive
Model of Psychological Resilience.''} \emph{Journal of Experimental
Psychopathology} 7 (3): 296--310.

\bibitem[\citeproctext]{ref-Romanyukha2023}
Romanyukha, Alexei Alexeevich, Konstantin Alexandrovich Novikov,
Konstantin Konstantinovich Avilov, Timofey Alexandrovich Nestik, and
Tatiana Evgenevna Sannikova. 2023. {``The Trade-Off Between COVID-19 and
Mental Diseases Burden During a Lockdown: Mathematical Modeling of
Control Measures.''} \emph{Infectious Disease Modelling} 8 (2): 403--14.
\url{https://doi.org/10.1016/j.idm.2023.04.003}.

\bibitem[\citeproctext]{ref-troy2023psychological}
Troy, Allison S, Emily C Willroth, Amanda J Shallcross, Nicole R
Giuliani, James J Gross, and Iris B Mauss. 2023. {``Psychological
Resilience: An Affect-Regulation Framework.''} \emph{Annual Review of
Psychology} 74 (1): 547--76.

\bibitem[\citeproctext]{ref-ZagerKocjan2021}
Zager Kocjan, Gaja, Tina Kavčič, and Andreja Avsec. 2021. {``Resilience
Matters: Explaining the Association Between Personality and
Psychological Functioning During the COVID-19 Pandemic.''}
\emph{International Journal of Clinical and Health Psychology} 21 (1):
100198. \url{https://doi.org/10.1016/j.ijchp.2020.08.002}.

\end{CSLReferences}




\end{document}
