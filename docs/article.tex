% Options for packages loaded elsewhere
% Options for packages loaded elsewhere
\PassOptionsToPackage{unicode}{hyperref}
\PassOptionsToPackage{hyphens}{url}
\PassOptionsToPackage{dvipsnames,svgnames,x11names}{xcolor}
%
\documentclass[
  letterpaper,
  DIV=11,
  numbers=noendperiod]{scrartcl}
\usepackage{xcolor}
\usepackage{amsmath,amssymb}
\setcounter{secnumdepth}{-\maxdimen} % remove section numbering
\usepackage{iftex}
\ifPDFTeX
  \usepackage[T1]{fontenc}
  \usepackage[utf8]{inputenc}
  \usepackage{textcomp} % provide euro and other symbols
\else % if luatex or xetex
  \usepackage{unicode-math} % this also loads fontspec
  \defaultfontfeatures{Scale=MatchLowercase}
  \defaultfontfeatures[\rmfamily]{Ligatures=TeX,Scale=1}
\fi
\usepackage{lmodern}
\ifPDFTeX\else
  % xetex/luatex font selection
\fi
% Use upquote if available, for straight quotes in verbatim environments
\IfFileExists{upquote.sty}{\usepackage{upquote}}{}
\IfFileExists{microtype.sty}{% use microtype if available
  \usepackage[]{microtype}
  \UseMicrotypeSet[protrusion]{basicmath} % disable protrusion for tt fonts
}{}
\makeatletter
\@ifundefined{KOMAClassName}{% if non-KOMA class
  \IfFileExists{parskip.sty}{%
    \usepackage{parskip}
  }{% else
    \setlength{\parindent}{0pt}
    \setlength{\parskip}{6pt plus 2pt minus 1pt}}
}{% if KOMA class
  \KOMAoptions{parskip=half}}
\makeatother
% Make \paragraph and \subparagraph free-standing
\makeatletter
\ifx\paragraph\undefined\else
  \let\oldparagraph\paragraph
  \renewcommand{\paragraph}{
    \@ifstar
      \xxxParagraphStar
      \xxxParagraphNoStar
  }
  \newcommand{\xxxParagraphStar}[1]{\oldparagraph*{#1}\mbox{}}
  \newcommand{\xxxParagraphNoStar}[1]{\oldparagraph{#1}\mbox{}}
\fi
\ifx\subparagraph\undefined\else
  \let\oldsubparagraph\subparagraph
  \renewcommand{\subparagraph}{
    \@ifstar
      \xxxSubParagraphStar
      \xxxSubParagraphNoStar
  }
  \newcommand{\xxxSubParagraphStar}[1]{\oldsubparagraph*{#1}\mbox{}}
  \newcommand{\xxxSubParagraphNoStar}[1]{\oldsubparagraph{#1}\mbox{}}
\fi
\makeatother


\usepackage{longtable,booktabs,array}
\usepackage{calc} % for calculating minipage widths
% Correct order of tables after \paragraph or \subparagraph
\usepackage{etoolbox}
\makeatletter
\patchcmd\longtable{\par}{\if@noskipsec\mbox{}\fi\par}{}{}
\makeatother
% Allow footnotes in longtable head/foot
\IfFileExists{footnotehyper.sty}{\usepackage{footnotehyper}}{\usepackage{footnote}}
\makesavenoteenv{longtable}
\usepackage{graphicx}
\makeatletter
\newsavebox\pandoc@box
\newcommand*\pandocbounded[1]{% scales image to fit in text height/width
  \sbox\pandoc@box{#1}%
  \Gscale@div\@tempa{\textheight}{\dimexpr\ht\pandoc@box+\dp\pandoc@box\relax}%
  \Gscale@div\@tempb{\linewidth}{\wd\pandoc@box}%
  \ifdim\@tempb\p@<\@tempa\p@\let\@tempa\@tempb\fi% select the smaller of both
  \ifdim\@tempa\p@<\p@\scalebox{\@tempa}{\usebox\pandoc@box}%
  \else\usebox{\pandoc@box}%
  \fi%
}
% Set default figure placement to htbp
\def\fps@figure{htbp}
\makeatother


% definitions for citeproc citations
\NewDocumentCommand\citeproctext{}{}
\NewDocumentCommand\citeproc{mm}{%
  \begingroup\def\citeproctext{#2}\cite{#1}\endgroup}
\makeatletter
 % allow citations to break across lines
 \let\@cite@ofmt\@firstofone
 % avoid brackets around text for \cite:
 \def\@biblabel#1{}
 \def\@cite#1#2{{#1\if@tempswa , #2\fi}}
\makeatother
\newlength{\cslhangindent}
\setlength{\cslhangindent}{1.5em}
\newlength{\csllabelwidth}
\setlength{\csllabelwidth}{3em}
\newenvironment{CSLReferences}[2] % #1 hanging-indent, #2 entry-spacing
 {\begin{list}{}{%
  \setlength{\itemindent}{0pt}
  \setlength{\leftmargin}{0pt}
  \setlength{\parsep}{0pt}
  % turn on hanging indent if param 1 is 1
  \ifodd #1
   \setlength{\leftmargin}{\cslhangindent}
   \setlength{\itemindent}{-1\cslhangindent}
  \fi
  % set entry spacing
  \setlength{\itemsep}{#2\baselineskip}}}
 {\end{list}}
\usepackage{calc}
\newcommand{\CSLBlock}[1]{\hfill\break\parbox[t]{\linewidth}{\strut\ignorespaces#1\strut}}
\newcommand{\CSLLeftMargin}[1]{\parbox[t]{\csllabelwidth}{\strut#1\strut}}
\newcommand{\CSLRightInline}[1]{\parbox[t]{\linewidth - \csllabelwidth}{\strut#1\strut}}
\newcommand{\CSLIndent}[1]{\hspace{\cslhangindent}#1}



\setlength{\emergencystretch}{3em} % prevent overfull lines

\providecommand{\tightlist}{%
  \setlength{\itemsep}{0pt}\setlength{\parskip}{0pt}}



 


\usepackage{longtable}
\usepackage{amsmath}
\usepackage{amssymb}
\usepackage{multicol}
\usepackage{float}
\usepackage{typearea}
\floatplacement{figure}{htbp}
\floatplacement{table}{htbp}
\AtBeginDocument{%
  \storeareas\normalpapersize
}
\BeforeRestoreareas{\cleardoublepage}
\newcommand*\uselandscape{%
  \cleardoublepage
  \KOMAoptions{paper=landscape}%
  \recalctypearea
  \areaset{1.2\textwidth}{1.2\textheight}%
}
\KOMAoption{captions}{tableheading}
\makeatletter
\@ifpackageloaded{caption}{}{\usepackage{caption}}
\AtBeginDocument{%
\ifdefined\contentsname
  \renewcommand*\contentsname{Table of contents}
\else
  \newcommand\contentsname{Table of contents}
\fi
\ifdefined\listfigurename
  \renewcommand*\listfigurename{List of Figures}
\else
  \newcommand\listfigurename{List of Figures}
\fi
\ifdefined\listtablename
  \renewcommand*\listtablename{List of Tables}
\else
  \newcommand\listtablename{List of Tables}
\fi
\ifdefined\figurename
  \renewcommand*\figurename{Figure}
\else
  \newcommand\figurename{Figure}
\fi
\ifdefined\tablename
  \renewcommand*\tablename{Table}
\else
  \newcommand\tablename{Table}
\fi
}
\@ifpackageloaded{float}{}{\usepackage{float}}
\floatstyle{ruled}
\@ifundefined{c@chapter}{\newfloat{codelisting}{h}{lop}}{\newfloat{codelisting}{h}{lop}[chapter]}
\floatname{codelisting}{Listing}
\newcommand*\listoflistings{\listof{codelisting}{List of Listings}}
\makeatother
\makeatletter
\makeatother
\makeatletter
\@ifpackageloaded{caption}{}{\usepackage{caption}}
\@ifpackageloaded{subcaption}{}{\usepackage{subcaption}}
\makeatother
\usepackage{bookmark}
\IfFileExists{xurl.sty}{\usepackage{xurl}}{} % add URL line breaks if available
\urlstyle{same}
\hypersetup{
  pdftitle={An agent-based simulation of psychological resilience},
  pdfauthor={Aly Lamuri},
  colorlinks=true,
  linkcolor={blue},
  filecolor={Maroon},
  citecolor={Blue},
  urlcolor={Blue},
  pdfcreator={LaTeX via pandoc}}


\title{An agent-based simulation of psychological resilience}
\author{Aly Lamuri}
\date{}
\begin{document}
\maketitle


\section{Introduction}\label{introduction}

Anxiety and depression are among the most prevalent psychological
disorders in the general population, generating substantial
socioeconomic consequences (GBD Collaborators 2022; Arias, Saxena, and
Verguet 2022). These conditions impose considerable direct and indirect
costs, while curative and rehabilitative approaches often demonstrate
limited effectiveness (Mekonen et al. 2021). Strengthening psychological
resilience offers a preventive pathway to reduce vulnerability to such
disorders (Troy et al. 2023). Public health policy should therefore
prioritize preventive strategies that foster resilience. However,
designing cost-effective preventive policies requires rigorous
assessment methods. Agent-based modeling (ABM) provides a promising
approach to simulate psychological resilience and generate evidence to
guide interventions aimed at enhancing resilience at the population
level.

\subsection{Previous ABM Publications}\label{previous-abm-publications}

The following five articles were selected for their relevance to the
development of ABM simulations addressing mental health, social
interaction, or their intersection. Each study contributed unique
methodological insights that informed the current work. While some
articles demonstrated only minimal overlap with our research, they were
included because they resolved specific modeling challenges pertinent to
our framework. Collectively, these studies provided a foundation for
constructing comprehensive and realistic ABM simulations in the context
of mental health and social dynamics.

Romanyukha et al. (2023) implemented an ABM to evaluate the mental
health consequences of policy measures during the COVID-19 pandemic. The
model integrated epidemic dynamics with the development of mental
disorders in a large urban population. Using Quality-Adjusted Life Years
(QALY) as an outcome measure, the study considered major depressive
disorder, anxiety disorder, COVID-19 cases (lethal and non-lethal), and
immunization. The results showed that mental disorders accounted for a
substantial share of pandemic-related health losses, and under strong
lockdown conditions, the lowest QALY loss occurred when approximately
70\% of the population was isolated. These findings demonstrate that ABM
can quantify the impact of policy interventions on mental health
outcomes, supporting its use for evaluating preventive strategies that
enhance psychological resilience.

Murase et al. (2021) introduced a generalized weighted social network
(GWSN) model to describe the formation of social ties and networks. The
GWSN model integrates mechanisms previously examined in isolation,
including triadic closure (the tendency for friends of friends to become
friends), homophilic interactions, and link termination processes. Due
to the model's complexity and numerous input parameters, the authors
combined extensive simulations with deep neural networks for regression
and global sensitivity analysis, enabling prediction of network
properties and identification of key determinants of network dynamics.
By providing a robust framework for modeling realistic social
interactions, this approach is directly relevant for simulating how
social connectivity influences resilience and the spread of mental
disorders in ABM.

Andarlia and Gunawan (2021) examined the contagion effect of
psychological disorders using ABM to simulate the spread of depression
in a population. The model accounted for the influence of close social
relationships, which could trigger mild, moderate, or severe depressive
episodes, and modeled recovery through therapy with transition dynamics
and recovery rates differentiated by gender. Simulation results
indicated that higher contact rates with depressed individuals increased
episode severity, whereas greater therapy uptake reduced depression
prevalence. This study highlights how ABM can capture both the
propagation and mitigation of mental disorders through social
mechanisms, a key component for modeling interventions aimed at
strengthening psychological resilience.

López and Giovanini (2024) introduced a methodology embedding an ABM
within adaptive temporal networks to simulate daily interactions and
examine the interplay between infectious disease spread and individual
behavior. The model integrates individual behavior, social dynamics, and
epidemiological factors, validated using real-world influenza outbreak
data, and can simulate complex social phenomena such as social
awareness. By incorporating self-organized system logic, agents respond
dynamically to external stimuli, perceptions, and health states.
Combined with social network models and mental health contagion
dynamics, this approach demonstrates that ABM can simulate emergent
phenomena arising from the interaction of social behavior and health,
providing a powerful tool for testing policies that influence population
resilience.

Benny et al. (2022) employed ABM to evaluate the impact of government
policies on mental health, focusing on depression in expectant mothers.
The model incorporated parameters from the ``All Our Families'' cohort
dataset and literature, including household income, age, education, and
environmental factors. The study hypothesized that income-supportive
programs would reduce maternal depression, and simulations confirmed
that progressive income policies significantly lowered prevalence.
Additionally, modeling mothers' social networks showed that expanding
social connections further reduced depression risk. These findings
illustrate that ABM can assess the effectiveness of policy interventions
on mental health outcomes, underscoring its applicability for designing
strategies to enhance resilience and reduce the prevalence of
psychological disorders.

\subsection{Current Research}\label{current-research}

This study aims to simulate stress perception using an ABM approach.
Stress perception is operationalized as a generalization of the
Perceived Stress Scale-10 (PSS-10), following the guidance provided in
its interpretation manual (Cohen 1988). Modeling stress perception is
essential for capturing the activation of psychological resilience
(Parsons, Kruijt, and Fox 2016). Under equivalent stress exposure,
agents with higher resilience are less likely to develop psychological
disorders, whereas less resilient agents are more vulnerable. The
simulation incorporates multiple psychological resources, including
social support, family support, and psychological capital, with the
latter encompassing personality traits such as openness,
conscientiousness, and extraversion (Hobfoll 1989; Klaver et al. 2021;
Zager Kocjan, Kavčič, and Avsec 2021).

\section{Methods}\label{methods}

\subsection{Affect}\label{affect}

Affect represented an individual's current emotional state, ranging from
very negative to very positive. The model examined how various factors
influenced these emotional states, including social connections, stress
events, and natural tendencies to return to an individual's baseline
emotional level.

The affect of agent \(i\) at time \(t+1\) was influenced by the affect
of neighboring agents \(j \in N_i\):

\[\Delta a_{i,t+1}^{social} = \alpha \sum_{j \in N_i} (a_{j,t} - a_{i,t})\]

where \(\alpha\) denoted the peer influence rate and \(N_i\) represented
the set of neighboring agents.

Stress events modified affect based on challenge--hindrance appraisal
and coping outcomes:

\[\Delta a_{i,t+1}^{stress} = \beta \cdot (c_i \cdot s_i - h_i \cdot f_i)\]

where \(\beta\) indicated the event appraisal rate, \(c_i\) and \(h_i\)
represented the challenge and hindrance components, respectively,
\(s_i\) denoted coping success (1 for success, 0 for failure), and
\(f_i\) denoted coping failure (1 for failure, 0 for success).

Affect naturally returned to baseline through homeostatic processes:

\[a_{i,t+1} = a_{i,t} + \gamma (a_{i,0} - a_{i,t}) + \Delta a_{i,t+1}^{social} + \Delta a_{i,t+1}^{stress}\]

where \(\gamma\) was the homeostatic rate and \(a_{i,0}\) represented
the baseline affect level.

Positive affect enhanced resource regeneration, whereas high resilience
provided emotional buffering:

\[r_{i,t+1} = r_{i,t} + \delta \cdot \max(0, a_{i,t}) \cdot R_{max} - \epsilon \cdot \max(0, -a_{i,t})\]

where \(r_i\) represented resources, \(\delta\) denoted the positive
affect regeneration bonus, \(\epsilon\) denoted the negative affect
regeneration penalty, and \(R_{max}\) represented the maximum resource
capacity.

\subsection{Stress Perception}\label{stress-perception}

Stress perception followed a comprehensive pipeline that transformed
life events into psychological stress responses using the
challenge--hindrance framework.

Each stress event \(e_i\) was characterized by controllability
\(c_i \in [0,1]\) and overload \(o_i \in [0,1]\):

\[z_i = \omega_c c_i - \omega_o o_i + b\]

\[challenge_i = \sigma(\gamma z_i) = \frac{1}{1 + e^{-\gamma z_i}}\]

\[hindrance_i = 1 - challenge_i\]

where \(\omega_c\) and \(\omega_o\) were weight parameters, \(b\) was a
bias term, and \(\gamma\) controlled the steepness of the sigmoid
function.

Controllability and overload were then modeled according to the bifactor
structure of the PSS-10:

\[PSS_{10_{total}} = \sum_{k=1}^{10} item_k\]

\[item_k = \mu_k + \lambda_k^c \cdot C + \lambda_k^o \cdot O + \epsilon_k\]

where \(C \sim N(0,1)\) and \(O \sim N(0,1)\) were latent factors with
correlation \(\rho\), \(\lambda_k^c\) and \(\lambda_k^o\) were factor
loadings, \(\mu_k\) represented item means, and \(\epsilon_k\) denoted
residuals.

An agent became stressed when the appraised stress load exceeded their
dynamic threshold:

\[L_i = s_i \cdot (1 + \delta \cdot (hindrance_i - challenge_i))\]

\[T_{i,effective} = T_{base} + \lambda_C \cdot challenge_i - \lambda_H \cdot hindrance_i\]

\[stressed_i = \begin{cases} 1 & \text{if } L_i > T_{i,effective} \\ 0 & \text{otherwise} \end{cases}\]

where \(s_i\) was the event magnitude, \(\delta\) controlled the
challenge--hindrance polarity, \(\lambda_C\) and \(\lambda_H\) were
threshold scaling parameters, and \(T_{base}\) represented the baseline
threshold.

\subsection{Resilience}\label{resilience}

Resilience represented an individual's capacity to adapt and recover
from stress events, with state variables including current resilience
level, baseline resilience, consecutive hindrances, and stress breach
count.

Resilience updated based on challenge--hindrance appraisal and coping
success:

\[\Delta r_i = \begin{cases}
\rho_{success} \cdot challenge_i & \text{if coping successful} \\
\rho_{minor} \cdot hindrance_i & \text{if coping successful} \\
-\rho_{failure} \cdot hindrance_i & \text{if coping failed} \\
-\rho_{minor} \cdot challenge_i & \text{if coping failed}
\end{cases}\]

where \(\rho_{success}\), \(\rho_{minor}\), and \(\rho_{failure}\) were
rate parameters corresponding to different outcomes.

Social support enhanced resilience:

\[r_{i,t+1} = r_{i,t} + \sigma_{support} \cdot \sum_{j \in supporters} q_j\]

where \(\sigma_{support}\) denoted the social support rate and \(q_j\)
represented the quality of support from agent \(j\).

Cumulative hindrance events generated overload when exceeding a
threshold:

\[overload_i = \begin{cases} 1 & \text{if } h_i^{consecutive} > \theta_{overload} \\ 0 & \text{otherwise} \end{cases}\]

\[\Delta r_{i,overload} = -\nu \cdot overload_i\]

where \(\theta_{overload}\) represented the overload threshold and
\(\nu\) denoted the overload penalty rate.

Resilience naturally returned to baseline over time:

\[r_{i,t+1} = r_{i,t} + \eta (r_{i,0} - r_{i,t}) + \Delta r_i + \Delta r_{i,overload}\]

where \(\eta\) was the homeostatic rate and \(r_{i,0}\) represented
baseline resilience.

The probability of successful coping integrated challenge--hindrance
effects and social influence:

\[p_{coping_i} = p_{base} + \phi_C \cdot challenge_i - \phi_H \cdot hindrance_i + \psi \cdot \bar{a}_{N_i}\]

where \(p_{base}\) denoted the base coping probability, \(\phi_C\) and
\(\phi_H\) were challenge and hindrance modifiers, \(\psi\) represented
the strength of social influence, and \(\bar{a}_{N_i}\) was the average
affect of neighboring agents.

\subsection{Agent Interaction}\label{agent-interaction}

Agent interactions formed the core social dynamics through a graph-based
small-world network generated using the Watts--Strogatz model.

Social interactions occurred randomly between adjacent nodes and
produced bidirectional affect changes with asymmetric effects:

\[\Delta a_{i,t+1} = \mu \sum_{j \in N_i} (a_{j,t} - a_{i,t})\]

\[\Delta a_{j,t+1} = \mu \sum_{i \in N_j} (a_{i,t} - a_{j,t})\]

where \(\mu\) denoted the interaction influence rate.

Stress triggered network rewiring based on homophily and support
effectiveness:

\[p_{rewire_i} = \begin{cases}
\pi & \text{if } stress_i^{breaches} > \theta_{adapt} \\ 0 & \text{otherwise} \end{cases}\]

Connection probability accounted for stress and affect similarity:

\[p_{connect_{ij}} = \frac{1}{1 + e^{-\kappa (similarity_{ij} - 0.5)}}\]

\[similarity_{ij} = 1 - \frac{|stress_i - stress_j| + |a_i - a_j|}{2}\]

where \(\kappa\) controlled the strength of homophily.

Agent interactions also reflected social support, depending on neighbor
affect and resilience:

\[support_{ij} = \begin{cases}
q_j \cdot r_j & \text{if } a_j > \theta_{support} \\ 0 & \text{otherwise} \end{cases}\]

where \(q_j\) represented support quality and \(\theta_{support}\)
denoted the support threshold.

\subsection{Psychological Resource
Management}\label{psychological-resource-management}

Resource management implemented the conservation of resources theory,
where individuals possessed limited psychological and physical resources
to allocate across protective strategies.

Resources regenerated toward maximum capacity, modulated by affect:

\[R_{i,t+1} = R_{i,t} + \rho (R_{max} - R_{i,t}) + \delta \cdot \max(0, a_{i,t}) \cdot R_{max}\]

where \(\rho\) denoted the base regeneration rate, \(R_{max}\)
represented the maximum resource capacity, and \(\delta\) was the
positive affect bonus.

Successful coping consumed resources proportional to the event
hindrance:

\[R_{i,t+1} = R_{i,t} - \kappa \cdot hindrance_i \cdot s_i\]

where \(\kappa\) represented the resource cost parameter and \(s_i\)
indicated coping success.

Resources were allocated across protective factors using a softmax
decision-making process:

\[allocation_{i,k} = \frac{e^{u_{i,k}/\tau}}{\sum_{m=1}^4 e^{u_{i,m}/\tau}} \cdot R_{i,t}\]

\[u_{i,k} = \alpha_k \cdot (1 - p_{i,k}) + \beta \cdot (1 - resilience_i)\]

where \(\tau\) denoted the softmax temperature, \(u_{i,k}\) represented
the utility of protective factor \(k\), \(\alpha_k\) indicated the
efficacy of factor \(k\), \(p_{i,k}\) was the current protection level,
and \(\beta\) controlled need-based allocation.

Protective factors improved based on resource investment:

\[\Delta p_{i,k} = \gamma_k \cdot allocation_{i,k} \cdot (1 - p_{i,k})\]

where \(\gamma_k\) represented the improvement rate for protective
factor \(k\).

\subsection{Model Integration}\label{model-integration}

The agent-based model integrated six core mechanism groups that worked
together to simulate realistic mental health dynamics through a
comprehensive state-transition framework.

Each agent \(i\) maintained a complete state vector that was updated
daily:

\[\mathbf{s}_{i,t+1} = f(\mathbf{s}_{i,t}, \mathbf{e}_{i,t}, \mathbf{n}_{i,t}, \mathbf{\theta})\]

where
\(\mathbf{s}_i = [a_i, r_i, stress_i, \mathbf{p}_i, R_i, pss10_i]\)
represented affect, resilience, stress level, protective factors,
resources, and PSS-10 scores; \(\mathbf{e}_i\) were external events;
\(\mathbf{n}_i\) were network influences; and \(\mathbf{\theta}\)
denoted model parameters.

Each simulation day followed a structured sequence:

\begin{enumerate}
\def\labelenumi{\arabic{enumi}.}
\tightlist
\item
  Initialization:
  \(\mathbf{s}_{i,t}^{init} = initialize(\mathbf{s}_{i,t-1})\)
\item
  Event Processing:
  \(\mathbf{e}_{i,t} \sim P(\mathbf{e}|\mathbf{s}_{i,t}^{init}, \mathbf{\theta})\)
\item
  Social Interaction:
  \(\Delta \mathbf{s}_{i,t}^{social} = g(\mathbf{s}_{N_i,t}, \mathbf{\theta}_{social})\)
\item
  Stress Appraisal:
  \(stress_{i,t} = appraise(\mathbf{e}_{i,t}, \mathbf{\theta}_{stress})\)
\item
  Coping Determination:
  \(coping_i = h(stress_{i,t}, r_{i,t}, \mathbf{s}_{N_i,t})\)
\item
  State Updates:
  \(\mathbf{s}_{i,t+1} = update(\mathbf{s}_{i,t}, \Delta \mathbf{s}_{i,t}^{social}, stress_{i,t}, coping_i)\)
\item
  Homeostasis:
  \(\mathbf{s}_{i,t+1} = homeostasis(\mathbf{s}_{i,t+1}, \mathbf{s}_{i,0})\)
\end{enumerate}

Positive reinforcement loops were represented as:

\[r_{t+1} = r_t + \phi_{success} \cdot challenge \cdot coping_{success}\]
\[a_{t+1} = a_t + \beta \cdot coping_{success} \cdot challenge\]
\[R_{t+1} = R_t + \delta \cdot \max(0, a_{t+1})\]

Negative degradation loops were represented as:

\[r_{t+1} = r_t - \phi_{failure} \cdot hindrance \cdot coping_{failure}\]
\[a_{t+1} = a_t - \beta \cdot coping_{failure} \cdot hindrance\]
\[R_{t+1} = R_t - \epsilon \cdot \max(0, -a_{t+1})\]

Social amplification loops were represented as:

\[\Delta a_i = \mu \sum_{j \in N_i} (a_j - a_i)\]
\[p_{connect_{ij}} = \frac{1}{1 + e^{-\kappa |a_i - a_j|}}\]

The system tended toward equilibrium points where all mechanisms
balanced:

\[\frac{d\mathbf{s}}{dt} = 0 \implies \mathbf{s}^* = equilibrium(\mathbf{\theta})\]

where \(\mathbf{s}^*\) represented stable mental health states,
depending on parameter configurations and the structure of the social
network.

\section{Results}\label{results}

Under homogeneous initial conditions representing a general population,
the model converged to a stable equilibrium across all key variables.
Mean perceived stress (\(\mu_{PSS-10}\) = 19.44 ± 0.10), resilience
(\(\mu_{res}\) = 0.48 ± 0.01), and affect (\(\mu_{affect}\) = −0.045 ±
0.004) remained consistent throughout the simulation. Variance across
agents was minimal, indicating that emotional and resilience homeostasis
mechanisms function as designed. Coping success stabilized around 0.45,
balancing challenge and hindrance appraisals. These findings validate
that the model captures stable psychological dynamics in the absence of
perturbation, forming a robust baseline for subsequent experiments
involving stress shocks or social disturbances.

\begin{longtable}[]{@{}
  >{\raggedright\arraybackslash}p{(\linewidth - 6\tabcolsep) * \real{0.3212}}
  >{\raggedright\arraybackslash}p{(\linewidth - 6\tabcolsep) * \real{0.2117}}
  >{\raggedright\arraybackslash}p{(\linewidth - 6\tabcolsep) * \real{0.1095}}
  >{\raggedright\arraybackslash}p{(\linewidth - 6\tabcolsep) * \real{0.3577}}@{}}
\caption{Descriptive findings on the model level}\tabularnewline
\toprule\noalign{}
\begin{minipage}[b]{\linewidth}\raggedright
\textbf{Description}
\end{minipage} & \begin{minipage}[b]{\linewidth}\raggedright
\textbf{Mean ± SD}
\end{minipage} & \begin{minipage}[b]{\linewidth}\raggedright
\textbf{Min -- Max}
\end{minipage} & \begin{minipage}[b]{\linewidth}\raggedright
\textbf{Interpretation}
\end{minipage} \\
\midrule\noalign{}
\endfirsthead
\toprule\noalign{}
\begin{minipage}[b]{\linewidth}\raggedright
\textbf{Description}
\end{minipage} & \begin{minipage}[b]{\linewidth}\raggedright
\textbf{Mean ± SD}
\end{minipage} & \begin{minipage}[b]{\linewidth}\raggedright
\textbf{Min -- Max}
\end{minipage} & \begin{minipage}[b]{\linewidth}\raggedright
\textbf{Interpretation}
\end{minipage} \\
\midrule\noalign{}
\endhead
\bottomrule\noalign{}
\endlastfoot
Mean perceived stress (PSS-10) across agents & 19.44 ± 0.10 & 19.08 --
19.76 & Stable population-level stress perception \\
Average resilience level & 0.478 ± 0.013 & 0.44 -- 0.51 & High
stability; minimal variance between agents \\
Mean affect (emotional valence) & −0.045 ± 0.004 & −0.055 -- −0.018 &
Neutral-negative baseline affect, low variability \\
Proportion of successful coping events & 0.446 ± 0.013 & 0.41 -- 0.49 &
Balanced coping outcomes (\textasciitilde45\% success) \\
Mean psychological resources & 0.111 ± 0.017 & 0.10 -- 0.47 & Narrowly
distributed, stable resource pool \\
Average current stress level & 0.651 ± 0.017 & 0.30 -- 0.69 & Mild,
steady stress levels across agents \\
Mean challenge vs.~hindrance appraisal & 0.500 ± 0.008 / 0.500 ± 0.008 &
0.47 -- 0.53 & Balanced stress appraisal pattern \\
Challenge--hindrance difference & 0.0006 ± 0.016 & −0.05 -- 0.05 &
Centered near zero, indicating equilibrium \\
Mean sequence of hindrance events & 1.86 ± 0.06 & 1.25 -- 2.04 &
Occasional stress streaks but stable overall \\
Count of successful coping events per agent & 679.7 ± 25.4 & 607 -- 759
& Moderate variability in coping outcomes \\
Number of social exchanges per agent & 1525.2 ± 38.6 & 1406 -- 1655 &
Consistent, dense social network activity \\
Number of support transactions & 167.2 ± 13.5 & 84 -- 215 & Regular but
bounded support flow \\
\end{longtable}

The model-level data reveal a system characterized by marked stability
and equilibrium across psychological and social dimensions. Aggregate
stress, resilience, and affect indicators exhibit limited variability,
suggesting that the simulated population reached a steady-state dynamic.
The balance between challenge and hindrance appraisals indicates that
agents, on average, perceived stressors as neither predominantly
threatening nor facilitating, implying a neutral cognitive framing of
stress events. Similarly, the near-constant network density and social
support rate highlight a static social structure in which interpersonal
connectivity remained fixed throughout the simulation. Despite this
structural stability, behavioral outcomes such as coping success and
social exchanges demonstrate bounded variability, reflecting modest
inter-agent heterogeneity in adaptive processes. Overall, the
model-level summary suggests that the collective system maintained
homeostasis while allowing limited individual differentiation in coping
and social behaviors.

\begin{longtable}[]{@{}
  >{\raggedright\arraybackslash}p{(\linewidth - 6\tabcolsep) * \real{0.3604}}
  >{\raggedright\arraybackslash}p{(\linewidth - 6\tabcolsep) * \real{0.1261}}
  >{\raggedright\arraybackslash}p{(\linewidth - 6\tabcolsep) * \real{0.1171}}
  >{\raggedright\arraybackslash}p{(\linewidth - 6\tabcolsep) * \real{0.3964}}@{}}
\caption{Descriptive findings on the agent level}\tabularnewline
\toprule\noalign{}
\begin{minipage}[b]{\linewidth}\raggedright
\textbf{Description}
\end{minipage} & \begin{minipage}[b]{\linewidth}\raggedright
\textbf{Mean ± SD}
\end{minipage} & \begin{minipage}[b]{\linewidth}\raggedright
\textbf{Min -- Max}
\end{minipage} & \begin{minipage}[b]{\linewidth}\raggedright
\textbf{Interpretation}
\end{minipage} \\
\midrule\noalign{}
\endfirsthead
\toprule\noalign{}
\begin{minipage}[b]{\linewidth}\raggedright
\textbf{Description}
\end{minipage} & \begin{minipage}[b]{\linewidth}\raggedright
\textbf{Mean ± SD}
\end{minipage} & \begin{minipage}[b]{\linewidth}\raggedright
\textbf{Min -- Max}
\end{minipage} & \begin{minipage}[b]{\linewidth}\raggedright
\textbf{Interpretation}
\end{minipage} \\
\midrule\noalign{}
\endhead
\bottomrule\noalign{}
\endlastfoot
Perceived stress (PSS-10) per agent-step & 19.44 ± 3.28 & 5 -- 35 &
Realistic spread in individual stress scores \\
Instantaneous resilience level & 0.478 ± 0.267 & 0.006 -- 1.00 & Wide
range; resilience varies across steps \\
Current emotional state & −0.045 ± 0.119 & −0.76 -- 0.53 & Normal
emotional fluctuation around neutral \\
Available psychological resources & 0.111 ± 0.071 & 0.05 -- 0.78 &
Resource regeneration and depletion balanced \\
Active stress intensity & 0.651 ± 0.344 & 0 -- 1.0 & Bounded stress
within coping limits \\
Perceived controllability of stress & 0.482 ± 0.128 & 0 -- 1.0 &
Moderate controllability on average \\
Perceived overload of stress events & 0.488 ± 0.105 & 0.04 -- 0.92 &
Balanced between manageable and high load \\
Ongoing hindrance streak length & 0.93 ± 1.32 & 0 -- 18.05 & Occasional
long stress streaks observed \\
\end{longtable}

At the agent level, the data depict dynamic variability embedded within
an overall stable population structure. Individual measures of stress,
affect, and resilience fluctuate widely across time and agents,
capturing the stochastic and adaptive nature of micro-level
psychological processes. The broad range in perceived stress and
resilience underscores individual differences in vulnerability and
coping potential, consistent with naturalistic behavioral diversity.
Measures of stress controllability and overload suggest a dynamic
tension between agents' perceived ability to manage stressors and their
exposure to them, resulting in a balance between effective adaptation
and temporary overload. The occurrence of extended hindrance sequences
further illustrates intermittent periods of cumulative stress before
recovery. Taken together, the agent-level summary portrays a complex
adaptive system in which individuals exhibit continuous fluctuations
around a stable population mean, thereby supporting emergent equilibrium
at the collective level.

\section{Discussion}\label{discussion}

The findings of this study demonstrate that the agent-based model
successfully reproduces stable psychological and social dynamics under
homogeneous baseline conditions. The convergence of key indicators,
including perceived stress, resilience, and affect, indicates that the
model reliably captures equilibrium states that align with empirical
expectations of population-level homeostasis. The observed stability
suggests that the implemented mechanisms governing stress perception,
coping, and social exchange are internally consistent and theoretically
coherent. This provides a solid foundation for further experimentation
involving environmental shocks or targeted interventions, where
deviations from equilibrium can reveal the resilience properties of the
simulated population.

At the macro level, the model demonstrates that collective patterns of
stress and resilience can remain balanced even when individual-level
fluctuations are present. The equilibrium between challenge and
hindrance appraisals implies that agents, on average, experience
stressors as manageable rather than destabilizing. This mirrors
empirical findings in resilience research, where moderate stress
exposure often promotes adaptive functioning rather than deterioration
(Fletcher and Sarkar 2013; Waldeck et al. 2021). The stability of
network characteristics, such as constant density and social exchange
rates, indicates that the system's social structure was sufficient to
sustain equilibrium without requiring additional external regulation.
Together, these results highlight the model's capacity to reproduce
homeostatic processes commonly observed in resilient populations.

At the micro level, individual trajectories exhibit substantial
variability, reflecting the stochastic and adaptive nature of
psychological processes within the model. Agents display diverse
patterns of stress perception, coping success, and affective regulation,
yet the system collectively stabilizes around its mean state. This
emergent property is consistent with theories of complex adaptive
systems, in which local heterogeneity supports global stability (Yan,
Martinez, and Liu 2017). The wide range in resilience and perceived
controllability further indicates that the model effectively captures
inter-individual differences that are central to resilience theory.
Importantly, transient stress overloads and recovery cycles emerge
naturally from the interaction of cognitive and social mechanisms,
underscoring the ecological validity of the model's behavioral
assumptions.

Overall, this study advances the use of agent-based modeling for
investigating psychological resilience and stress adaptation. By
reproducing both stability and variability, the model bridges
individual-level dynamics and population-level outcomes, offering a
valuable tool for testing preventive mental health interventions. Future
research should extend this framework by introducing perturbations such
as social disruptions, environmental stressors, or policy interventions
to examine how resilience unfolds under adversity. Incorporating
longitudinal empirical data would further strengthen the model's
external validity and support its application in the design of
resilience-enhancing public health strategies.

\section{Conclusion}\label{conclusion}

This study demonstrates the potential of agent-based modeling to capture
the dynamic interplay between stress perception, resilience, and social
interaction within a simulated population. By operationalizing
psychological constructs such as perceived stress, coping, and social
support, the model reproduces realistic patterns of equilibrium and
individual variability, consistent with the established theory on
resilience and adaptation. The results validate the internal coherence
of the model and establish a baseline framework for future experiments
involving perturbations or policy interventions. As such, this work
contributes a methodological foundation for the quantitative evaluation
of preventive mental health strategies, enabling systematic exploration
of how psychological and social mechanisms jointly shape population
resilience.

\section*{References}\label{references}
\addcontentsline{toc}{section}{References}

\multicols{2}
\footnotesize

\phantomsection\label{refs}
\begin{CSLReferences}{1}{0}
\bibitem[\citeproctext]{ref-Andarlia2021}
Andarlia, H T, and I Gunawan. 2021. {``An Agent-Based Model of Contagion
Effects in Affected Depression and Its Recovery Process.''}
\emph{Journal of Physics: Conference Series} 1751 (1): 012007.
\url{https://doi.org/10.1088/1742-6596/1751/1/012007}.

\bibitem[\citeproctext]{ref-arias2022quantifying}
Arias, Daniel, Shekhar Saxena, and Stéphane Verguet. 2022.
{``Quantifying the Global Burden of Mental Disorders and Their Economic
Value.''} \emph{EClinicalMedicine} 54.

\bibitem[\citeproctext]{ref-Benny2022}
Benny, Claire, Shelby Yamamoto, Sheila McDonald, Radha Chari, and Roman
Pabayo. 2022. {``Modelling Maternal Depression: An Agent-Based Model to
Examine the Complex Relationship Between Relative Income and
Depression.''} \emph{International Journal of Environmental Research and
Public Health} 19 (7): 4208.
\url{https://doi.org/10.3390/ijerph19074208}.

\bibitem[\citeproctext]{ref-cohen1988perceived}
Cohen, Sheldon. 1988. {``Perceived Stress in a Probability Sample of the
United States.''}

\bibitem[\citeproctext]{ref-Fletcher2013}
Fletcher, David, and Mustafa Sarkar. 2013. {``Psychological Resilience:
A Review and Critique of Definitions, Concepts, and Theory.''}
\emph{European Psychologist} 18 (1): 12--23.
\url{https://doi.org/10.1027/1016-9040/a000124}.

\bibitem[\citeproctext]{ref-gbd2022global}
GBD Collaborators. 2022. {``Global, Regional, and National Burden of 12
Mental Disorders in 204 Countries and Territories, 1990--2019: A
Systematic Analysis for the Global Burden of Disease Study 2019.''}
\emph{The Lancet Psychiatry} 9 (2): 137--50.

\bibitem[\citeproctext]{ref-Hobfoll1989}
Hobfoll, Stevan E. 1989. {``Conservation of Resources: A New Attempt at
Conceptualizing Stress.''} \emph{American Psychologist} 44 (3): 513--24.
\url{https://doi.org/10.1037/0003-066x.44.3.513}.

\bibitem[\citeproctext]{ref-klaver2021exposure}
Klaver, M, BJ Van den Hoofdakker, H Wouters, G De Kuijper, PJ Hoekstra,
and A De Bildt. 2021. {``Exposure to Challenging Behaviours and Burnout
Symptoms Among Care Staff: The Role of Psychological Resources.''}
\emph{Journal of Intellectual Disability Research} 65 (2): 173--85.

\bibitem[\citeproctext]{ref-Lopez2024}
López, Leonardo, and Leonardo Giovanini. 2024. {``Adaptive Dynamic
Social Networks Using an Agent-Based Model to Study the Role of Social
Awareness in Infectious Disease Spread,''} July.
\url{https://doi.org/10.1101/2024.07.16.24310475}.

\bibitem[\citeproctext]{ref-mekonen2021estimating}
Mekonen, Tesfa, Gary CK Chan, Jason P Connor, Leanne Hides, and Janni
Leung. 2021. {``Estimating the Global Treatment Rates for Depression: A
Systematic Review and Meta-Analysis.''} \emph{Journal of Affective
Disorders} 295: 1234--42.

\bibitem[\citeproctext]{ref-Murase2021}
Murase, Yohsuke, Hang-Hyun Jo, János Török, János Kertész, and Kimmo
Kaski. 2021. {``Deep Learning Exploration of Agent-Based Social Network
Model Parameters.''} \emph{Frontiers in Big Data} 4 (September).
\url{https://doi.org/10.3389/fdata.2021.739081}.

\bibitem[\citeproctext]{ref-parsons2016cognitive}
Parsons, Sam, Anne-Wil Kruijt, and Elaine Fox. 2016. {``A Cognitive
Model of Psychological Resilience.''} \emph{Journal of Experimental
Psychopathology} 7 (3): 296--310.

\bibitem[\citeproctext]{ref-Romanyukha2023}
Romanyukha, Alexei Alexeevich, Konstantin Alexandrovich Novikov,
Konstantin Konstantinovich Avilov, Timofey Alexandrovich Nestik, and
Tatiana Evgenevna Sannikova. 2023. {``The Trade-Off Between COVID-19 and
Mental Diseases Burden During a Lockdown: Mathematical Modeling of
Control Measures.''} \emph{Infectious Disease Modelling} 8 (2): 403--14.
\url{https://doi.org/10.1016/j.idm.2023.04.003}.

\bibitem[\citeproctext]{ref-troy2023psychological}
Troy, Allison S, Emily C Willroth, Amanda J Shallcross, Nicole R
Giuliani, James J Gross, and Iris B Mauss. 2023. {``Psychological
Resilience: An Affect-Regulation Framework.''} \emph{Annual Review of
Psychology} 74 (1): 547--76.

\bibitem[\citeproctext]{ref-Waldeck2021}
Waldeck, Daniel, Luca Pancani, Andrew Holliman, Maria Karekla, and Ian
Tyndall. 2021. {``Adaptability and Psychological Flexibility:
Overlapping Constructs?''} \emph{Journal of Contextual Behavioral
Science} 19 (January): 72--78.
\url{https://doi.org/10.1016/j.jcbs.2021.01.002}.

\bibitem[\citeproctext]{ref-Yan2017}
Yan, Gang, Neo D. Martinez, and Yang-Yu Liu. 2017. {``Degree
Heterogeneity and Stability of Ecological Networks.''} \emph{Journal of
The Royal Society Interface} 14 (131): 20170189.
\url{https://doi.org/10.1098/rsif.2017.0189}.

\bibitem[\citeproctext]{ref-ZagerKocjan2021}
Zager Kocjan, Gaja, Tina Kavčič, and Andreja Avsec. 2021. {``Resilience
Matters: Explaining the Association Between Personality and
Psychological Functioning During the COVID-19 Pandemic.''}
\emph{International Journal of Clinical and Health Psychology} 21 (1):
100198. \url{https://doi.org/10.1016/j.ijchp.2020.08.002}.

\end{CSLReferences}




\end{document}
