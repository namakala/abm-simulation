% Options for packages loaded elsewhere
\PassOptionsToPackage{unicode}{hyperref}
\PassOptionsToPackage{hyphens}{url}
\PassOptionsToPackage{dvipsnames,svgnames,x11names}{xcolor}
%
\documentclass[
  letterpaper,
  DIV=11,
  numbers=noendperiod]{scrartcl}

\usepackage{amsmath,amssymb}
\usepackage{iftex}
\ifPDFTeX
  \usepackage[T1]{fontenc}
  \usepackage[utf8]{inputenc}
  \usepackage{textcomp} % provide euro and other symbols
\else % if luatex or xetex
  \usepackage{unicode-math}
  \defaultfontfeatures{Scale=MatchLowercase}
  \defaultfontfeatures[\rmfamily]{Ligatures=TeX,Scale=1}
\fi
\usepackage{lmodern}
\ifPDFTeX\else  
    % xetex/luatex font selection
\fi
% Use upquote if available, for straight quotes in verbatim environments
\IfFileExists{upquote.sty}{\usepackage{upquote}}{}
\IfFileExists{microtype.sty}{% use microtype if available
  \usepackage[]{microtype}
  \UseMicrotypeSet[protrusion]{basicmath} % disable protrusion for tt fonts
}{}
\makeatletter
\@ifundefined{KOMAClassName}{% if non-KOMA class
  \IfFileExists{parskip.sty}{%
    \usepackage{parskip}
  }{% else
    \setlength{\parindent}{0pt}
    \setlength{\parskip}{6pt plus 2pt minus 1pt}}
}{% if KOMA class
  \KOMAoptions{parskip=half}}
\makeatother
\usepackage{xcolor}
\setlength{\emergencystretch}{3em} % prevent overfull lines
\setcounter{secnumdepth}{-\maxdimen} % remove section numbering
% Make \paragraph and \subparagraph free-standing
\makeatletter
\ifx\paragraph\undefined\else
  \let\oldparagraph\paragraph
  \renewcommand{\paragraph}{
    \@ifstar
      \xxxParagraphStar
      \xxxParagraphNoStar
  }
  \newcommand{\xxxParagraphStar}[1]{\oldparagraph*{#1}\mbox{}}
  \newcommand{\xxxParagraphNoStar}[1]{\oldparagraph{#1}\mbox{}}
\fi
\ifx\subparagraph\undefined\else
  \let\oldsubparagraph\subparagraph
  \renewcommand{\subparagraph}{
    \@ifstar
      \xxxSubParagraphStar
      \xxxSubParagraphNoStar
  }
  \newcommand{\xxxSubParagraphStar}[1]{\oldsubparagraph*{#1}\mbox{}}
  \newcommand{\xxxSubParagraphNoStar}[1]{\oldsubparagraph{#1}\mbox{}}
\fi
\makeatother


\providecommand{\tightlist}{%
  \setlength{\itemsep}{0pt}\setlength{\parskip}{0pt}}\usepackage{longtable,booktabs,array}
\usepackage{calc} % for calculating minipage widths
% Correct order of tables after \paragraph or \subparagraph
\usepackage{etoolbox}
\makeatletter
\patchcmd\longtable{\par}{\if@noskipsec\mbox{}\fi\par}{}{}
\makeatother
% Allow footnotes in longtable head/foot
\IfFileExists{footnotehyper.sty}{\usepackage{footnotehyper}}{\usepackage{footnote}}
\makesavenoteenv{longtable}
\usepackage{graphicx}
\makeatletter
\def\maxwidth{\ifdim\Gin@nat@width>\linewidth\linewidth\else\Gin@nat@width\fi}
\def\maxheight{\ifdim\Gin@nat@height>\textheight\textheight\else\Gin@nat@height\fi}
\makeatother
% Scale images if necessary, so that they will not overflow the page
% margins by default, and it is still possible to overwrite the defaults
% using explicit options in \includegraphics[width, height, ...]{}
\setkeys{Gin}{width=\maxwidth,height=\maxheight,keepaspectratio}
% Set default figure placement to htbp
\makeatletter
\def\fps@figure{htbp}
\makeatother
% definitions for citeproc citations
\NewDocumentCommand\citeproctext{}{}
\NewDocumentCommand\citeproc{mm}{%
  \begingroup\def\citeproctext{#2}\cite{#1}\endgroup}
\makeatletter
 % allow citations to break across lines
 \let\@cite@ofmt\@firstofone
 % avoid brackets around text for \cite:
 \def\@biblabel#1{}
 \def\@cite#1#2{{#1\if@tempswa , #2\fi}}
\makeatother
\newlength{\cslhangindent}
\setlength{\cslhangindent}{1.5em}
\newlength{\csllabelwidth}
\setlength{\csllabelwidth}{3em}
\newenvironment{CSLReferences}[2] % #1 hanging-indent, #2 entry-spacing
 {\begin{list}{}{%
  \setlength{\itemindent}{0pt}
  \setlength{\leftmargin}{0pt}
  \setlength{\parsep}{0pt}
  % turn on hanging indent if param 1 is 1
  \ifodd #1
   \setlength{\leftmargin}{\cslhangindent}
   \setlength{\itemindent}{-1\cslhangindent}
  \fi
  % set entry spacing
  \setlength{\itemsep}{#2\baselineskip}}}
 {\end{list}}
\usepackage{calc}
\newcommand{\CSLBlock}[1]{\hfill\break\parbox[t]{\linewidth}{\strut\ignorespaces#1\strut}}
\newcommand{\CSLLeftMargin}[1]{\parbox[t]{\csllabelwidth}{\strut#1\strut}}
\newcommand{\CSLRightInline}[1]{\parbox[t]{\linewidth - \csllabelwidth}{\strut#1\strut}}
\newcommand{\CSLIndent}[1]{\hspace{\cslhangindent}#1}

\usepackage{longtable}
\usepackage{amsmath}
\usepackage{amssymb}
\usepackage{multicol}
\usepackage{float}
\usepackage{typearea}
\floatplacement{figure}{htbp}
\floatplacement{table}{htbp}
\AtBeginDocument{%
  \storeareas\normalpapersize
}
\BeforeRestoreareas{\cleardoublepage}
\newcommand*\uselandscape{%
  \cleardoublepage
  \KOMAoptions{paper=landscape}%
  \recalctypearea
  \areaset{1.2\textwidth}{1.2\textheight}%
}
\KOMAoption{captions}{tableheading}
\makeatletter
\@ifpackageloaded{caption}{}{\usepackage{caption}}
\AtBeginDocument{%
\ifdefined\contentsname
  \renewcommand*\contentsname{Table of contents}
\else
  \newcommand\contentsname{Table of contents}
\fi
\ifdefined\listfigurename
  \renewcommand*\listfigurename{List of Figures}
\else
  \newcommand\listfigurename{List of Figures}
\fi
\ifdefined\listtablename
  \renewcommand*\listtablename{List of Tables}
\else
  \newcommand\listtablename{List of Tables}
\fi
\ifdefined\figurename
  \renewcommand*\figurename{Figure}
\else
  \newcommand\figurename{Figure}
\fi
\ifdefined\tablename
  \renewcommand*\tablename{Table}
\else
  \newcommand\tablename{Table}
\fi
}
\@ifpackageloaded{float}{}{\usepackage{float}}
\floatstyle{ruled}
\@ifundefined{c@chapter}{\newfloat{codelisting}{h}{lop}}{\newfloat{codelisting}{h}{lop}[chapter]}
\floatname{codelisting}{Listing}
\newcommand*\listoflistings{\listof{codelisting}{List of Listings}}
\makeatother
\makeatletter
\makeatother
\makeatletter
\@ifpackageloaded{caption}{}{\usepackage{caption}}
\@ifpackageloaded{subcaption}{}{\usepackage{subcaption}}
\makeatother

\ifLuaTeX
  \usepackage{selnolig}  % disable illegal ligatures
\fi
\usepackage{bookmark}

\IfFileExists{xurl.sty}{\usepackage{xurl}}{} % add URL line breaks if available
\urlstyle{same} % disable monospaced font for URLs
\hypersetup{
  pdftitle={An agent-based simulation of psychological resilience},
  pdfauthor={Aly Lamuri},
  colorlinks=true,
  linkcolor={blue},
  filecolor={Maroon},
  citecolor={Blue},
  urlcolor={Blue},
  pdfcreator={LaTeX via pandoc}}


\title{An agent-based simulation of psychological resilience}
\author{Aly Lamuri}
\date{}

\begin{document}
\maketitle


\section{Introduction}\label{introduction}

Anxiety and depression are among the most prevalent psychological
disorders in the general population, generating substantial
socioeconomic consequences (GBD Collaborators 2022; Arias, Saxena, and
Verguet 2022). These conditions impose considerable direct and indirect
costs, while curative and rehabilitative approaches often demonstrate
limited effectiveness (Mekonen et al. 2021). Strengthening psychological
resilience offers a preventive pathway to reduce vulnerability to such
disorders (Troy et al. 2023). Public health policy should therefore
prioritize preventive strategies that foster resilience. However,
designing cost-effective preventive policies requires rigorous
assessment methods. Agent-based modeling (ABM) provides a promising
approach to simulate psychological resilience and generate evidence to
guide interventions aimed at enhancing resilience at the population
level.

\subsection{Previous ABM Publications}\label{previous-abm-publications}

The following five articles were selected for their relevance to the
development of ABM simulations addressing mental health, social
interaction, or their intersection. Each study contributed unique
methodological insights that informed the current work. While some
articles demonstrated only minimal overlap with our research, they were
included because they resolved specific modeling challenges pertinent to
our framework. Collectively, these studies provided a foundation for
constructing comprehensive and realistic ABM simulations in the context
of mental health and social dynamics.

Romanyukha et al. (2023) implemented an ABM to evaluate the mental
health consequences of policy measures during the COVID-19 pandemic. The
model integrated epidemic dynamics with the development of mental
disorders in a large urban population. Using Quality-Adjusted Life Years
(QALY) as an outcome measure, the study considered major depressive
disorder, anxiety disorder, COVID-19 cases (lethal and non-lethal), and
immunization. The results showed that mental disorders accounted for a
substantial share of pandemic-related health losses, and under strong
lockdown conditions, the lowest QALY loss occurred when approximately
70\% of the population was isolated. These findings demonstrate that ABM
can quantify the impact of policy interventions on mental health
outcomes, supporting its use for evaluating preventive strategies that
enhance psychological resilience.

Murase et al. (2021) introduced a generalized weighted social network
(GWSN) model to describe the formation of social ties and networks. The
GWSN model integrates mechanisms previously examined in isolation,
including triadic closure (the tendency for friends of friends to become
friends), homophilic interactions, and link termination processes. Due
to the model's complexity and numerous input parameters, the authors
combined extensive simulations with deep neural networks for regression
and global sensitivity analysis, enabling prediction of network
properties and identification of key determinants of network dynamics.
By providing a robust framework for modeling realistic social
interactions, this approach is directly relevant for simulating how
social connectivity influences resilience and the spread of mental
disorders in ABM.

Andarlia and Gunawan (2021) examined the contagion effect of
psychological disorders using ABM to simulate the spread of depression
in a population. The model accounted for the influence of close social
relationships, which could trigger mild, moderate, or severe depressive
episodes, and modeled recovery through therapy with transition dynamics
and recovery rates differentiated by gender. Simulation results
indicated that higher contact rates with depressed individuals increased
episode severity, whereas greater therapy uptake reduced depression
prevalence. This study highlights how ABM can capture both the
propagation and mitigation of mental disorders through social
mechanisms, a key component for modeling interventions aimed at
strengthening psychological resilience.

López and Giovanini (2024) introduced a methodology embedding an ABM
within adaptive temporal networks to simulate daily interactions and
examine the interplay between infectious disease spread and individual
behavior. The model integrates individual behavior, social dynamics, and
epidemiological factors, validated using real-world influenza outbreak
data, and can simulate complex social phenomena such as social
awareness. By incorporating self-organized system logic, agents respond
dynamically to external stimuli, perceptions, and health states.
Combined with social network models and mental health contagion
dynamics, this approach demonstrates that ABM can simulate emergent
phenomena arising from the interaction of social behavior and health,
providing a powerful tool for testing policies that influence population
resilience.

Benny et al. (2022) employed ABM to evaluate the impact of government
policies on mental health, focusing on depression in expectant mothers.
The model incorporated parameters from the ``All Our Families'' cohort
dataset and literature, including household income, age, education, and
environmental factors. The study hypothesized that income-supportive
programs would reduce maternal depression, and simulations confirmed
that progressive income policies significantly lowered prevalence.
Additionally, modeling mothers' social networks showed that expanding
social connections further reduced depression risk. These findings
illustrate that ABM can assess the effectiveness of policy interventions
on mental health outcomes, underscoring its applicability for designing
strategies to enhance resilience and reduce the prevalence of
psychological disorders.

\subsection{Current Research}\label{current-research}

This study aims to simulate stress perception using an ABM approach.
Stress perception is operationalized as a generalization of the
Perceived Stress Scale-10, following the guidance provided in its
interpretation manual (Cohen 1988). Modeling stress perception is
essential for capturing the activation of psychological resilience
(Parsons, Kruijt, and Fox 2016). Under equivalent stress exposure,
agents with higher resilience are less likely to develop psychological
disorders, whereas less resilient agents are more vulnerable. The
simulation incorporates multiple psychological resources, including
social support, family support, and psychological capital, with the
latter encompassing personality traits such as openness,
conscientiousness, and extraversion (Hobfoll 1989; Klaver et al. 2021;
Zager Kocjan, Kavčič, and Avsec 2021).

\section*{References}\label{references}
\addcontentsline{toc}{section}{References}

\multicols{2}
\footnotesize

\phantomsection\label{refs}
\begin{CSLReferences}{1}{0}
\bibitem[\citeproctext]{ref-Andarlia2021}
Andarlia, H T, and I Gunawan. 2021. {``An Agent-Based Model of Contagion
Effects in Affected Depression and Its Recovery Process.''}
\emph{Journal of Physics: Conference Series} 1751 (1): 012007.
\url{https://doi.org/10.1088/1742-6596/1751/1/012007}.

\bibitem[\citeproctext]{ref-arias2022quantifying}
Arias, Daniel, Shekhar Saxena, and Stéphane Verguet. 2022.
{``Quantifying the Global Burden of Mental Disorders and Their Economic
Value.''} \emph{EClinicalMedicine} 54.

\bibitem[\citeproctext]{ref-Benny2022}
Benny, Claire, Shelby Yamamoto, Sheila McDonald, Radha Chari, and Roman
Pabayo. 2022. {``Modelling Maternal Depression: An Agent-Based Model to
Examine the Complex Relationship Between Relative Income and
Depression.''} \emph{International Journal of Environmental Research and
Public Health} 19 (7): 4208.
\url{https://doi.org/10.3390/ijerph19074208}.

\bibitem[\citeproctext]{ref-cohen1988perceived}
Cohen, Sheldon. 1988. {``Perceived Stress in a Probability Sample of the
United States.''}

\bibitem[\citeproctext]{ref-gbd2022global}
GBD Collaborators. 2022. {``Global, Regional, and National Burden of 12
Mental Disorders in 204 Countries and Territories, 1990--2019: A
Systematic Analysis for the Global Burden of Disease Study 2019.''}
\emph{The Lancet Psychiatry} 9 (2): 137--50.

\bibitem[\citeproctext]{ref-Hobfoll1989}
Hobfoll, Stevan E. 1989. {``Conservation of Resources: A New Attempt at
Conceptualizing Stress.''} \emph{American Psychologist} 44 (3): 513--24.
\url{https://doi.org/10.1037/0003-066x.44.3.513}.

\bibitem[\citeproctext]{ref-klaver2021exposure}
Klaver, M, BJ Van den Hoofdakker, H Wouters, G De Kuijper, PJ Hoekstra,
and A De Bildt. 2021. {``Exposure to Challenging Behaviours and Burnout
Symptoms Among Care Staff: The Role of Psychological Resources.''}
\emph{Journal of Intellectual Disability Research} 65 (2): 173--85.

\bibitem[\citeproctext]{ref-Lopez2024}
López, Leonardo, and Leonardo Giovanini. 2024. {``Adaptive Dynamic
Social Networks Using an Agent-Based Model to Study the Role of Social
Awareness in Infectious Disease Spread,''} July.
\url{https://doi.org/10.1101/2024.07.16.24310475}.

\bibitem[\citeproctext]{ref-mekonen2021estimating}
Mekonen, Tesfa, Gary CK Chan, Jason P Connor, Leanne Hides, and Janni
Leung. 2021. {``Estimating the Global Treatment Rates for Depression: A
Systematic Review and Meta-Analysis.''} \emph{Journal of Affective
Disorders} 295: 1234--42.

\bibitem[\citeproctext]{ref-Murase2021}
Murase, Yohsuke, Hang-Hyun Jo, János Török, János Kertész, and Kimmo
Kaski. 2021. {``Deep Learning Exploration of Agent-Based Social Network
Model Parameters.''} \emph{Frontiers in Big Data} 4 (September).
\url{https://doi.org/10.3389/fdata.2021.739081}.

\bibitem[\citeproctext]{ref-parsons2016cognitive}
Parsons, Sam, Anne-Wil Kruijt, and Elaine Fox. 2016. {``A Cognitive
Model of Psychological Resilience.''} \emph{Journal of Experimental
Psychopathology} 7 (3): 296--310.

\bibitem[\citeproctext]{ref-Romanyukha2023}
Romanyukha, Alexei Alexeevich, Konstantin Alexandrovich Novikov,
Konstantin Konstantinovich Avilov, Timofey Alexandrovich Nestik, and
Tatiana Evgenevna Sannikova. 2023. {``The Trade-Off Between COVID-19 and
Mental Diseases Burden During a Lockdown: Mathematical Modeling of
Control Measures.''} \emph{Infectious Disease Modelling} 8 (2): 403--14.
\url{https://doi.org/10.1016/j.idm.2023.04.003}.

\bibitem[\citeproctext]{ref-troy2023psychological}
Troy, Allison S, Emily C Willroth, Amanda J Shallcross, Nicole R
Giuliani, James J Gross, and Iris B Mauss. 2023. {``Psychological
Resilience: An Affect-Regulation Framework.''} \emph{Annual Review of
Psychology} 74 (1): 547--76.

\bibitem[\citeproctext]{ref-ZagerKocjan2021}
Zager Kocjan, Gaja, Tina Kavčič, and Andreja Avsec. 2021. {``Resilience
Matters: Explaining the Association Between Personality and
Psychological Functioning During the COVID-19 Pandemic.''}
\emph{International Journal of Clinical and Health Psychology} 21 (1):
100198. \url{https://doi.org/10.1016/j.ijchp.2020.08.002}.

\end{CSLReferences}




\end{document}
