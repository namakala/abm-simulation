% Options for packages loaded elsewhere
% Options for packages loaded elsewhere
\PassOptionsToPackage{unicode}{hyperref}
\PassOptionsToPackage{hyphens}{url}
\PassOptionsToPackage{dvipsnames,svgnames,x11names}{xcolor}
%
\documentclass[
  letterpaper,
  DIV=11,
  numbers=noendperiod]{scrartcl}
\usepackage{xcolor}
\usepackage{amsmath,amssymb}
\setcounter{secnumdepth}{-\maxdimen} % remove section numbering
\usepackage{iftex}
\ifPDFTeX
  \usepackage[T1]{fontenc}
  \usepackage[utf8]{inputenc}
  \usepackage{textcomp} % provide euro and other symbols
\else % if luatex or xetex
  \usepackage{unicode-math} % this also loads fontspec
  \defaultfontfeatures{Scale=MatchLowercase}
  \defaultfontfeatures[\rmfamily]{Ligatures=TeX,Scale=1}
\fi
\usepackage{lmodern}
\ifPDFTeX\else
  % xetex/luatex font selection
\fi
% Use upquote if available, for straight quotes in verbatim environments
\IfFileExists{upquote.sty}{\usepackage{upquote}}{}
\IfFileExists{microtype.sty}{% use microtype if available
  \usepackage[]{microtype}
  \UseMicrotypeSet[protrusion]{basicmath} % disable protrusion for tt fonts
}{}
\makeatletter
\@ifundefined{KOMAClassName}{% if non-KOMA class
  \IfFileExists{parskip.sty}{%
    \usepackage{parskip}
  }{% else
    \setlength{\parindent}{0pt}
    \setlength{\parskip}{6pt plus 2pt minus 1pt}}
}{% if KOMA class
  \KOMAoptions{parskip=half}}
\makeatother
% Make \paragraph and \subparagraph free-standing
\makeatletter
\ifx\paragraph\undefined\else
  \let\oldparagraph\paragraph
  \renewcommand{\paragraph}{
    \@ifstar
      \xxxParagraphStar
      \xxxParagraphNoStar
  }
  \newcommand{\xxxParagraphStar}[1]{\oldparagraph*{#1}\mbox{}}
  \newcommand{\xxxParagraphNoStar}[1]{\oldparagraph{#1}\mbox{}}
\fi
\ifx\subparagraph\undefined\else
  \let\oldsubparagraph\subparagraph
  \renewcommand{\subparagraph}{
    \@ifstar
      \xxxSubParagraphStar
      \xxxSubParagraphNoStar
  }
  \newcommand{\xxxSubParagraphStar}[1]{\oldsubparagraph*{#1}\mbox{}}
  \newcommand{\xxxSubParagraphNoStar}[1]{\oldsubparagraph{#1}\mbox{}}
\fi
\makeatother


\usepackage{longtable,booktabs,array}
\usepackage{calc} % for calculating minipage widths
% Correct order of tables after \paragraph or \subparagraph
\usepackage{etoolbox}
\makeatletter
\patchcmd\longtable{\par}{\if@noskipsec\mbox{}\fi\par}{}{}
\makeatother
% Allow footnotes in longtable head/foot
\IfFileExists{footnotehyper.sty}{\usepackage{footnotehyper}}{\usepackage{footnote}}
\makesavenoteenv{longtable}
\usepackage{graphicx}
\makeatletter
\newsavebox\pandoc@box
\newcommand*\pandocbounded[1]{% scales image to fit in text height/width
  \sbox\pandoc@box{#1}%
  \Gscale@div\@tempa{\textheight}{\dimexpr\ht\pandoc@box+\dp\pandoc@box\relax}%
  \Gscale@div\@tempb{\linewidth}{\wd\pandoc@box}%
  \ifdim\@tempb\p@<\@tempa\p@\let\@tempa\@tempb\fi% select the smaller of both
  \ifdim\@tempa\p@<\p@\scalebox{\@tempa}{\usebox\pandoc@box}%
  \else\usebox{\pandoc@box}%
  \fi%
}
% Set default figure placement to htbp
\def\fps@figure{htbp}
\makeatother


% definitions for citeproc citations
\NewDocumentCommand\citeproctext{}{}
\NewDocumentCommand\citeproc{mm}{%
  \begingroup\def\citeproctext{#2}\cite{#1}\endgroup}
\makeatletter
 % allow citations to break across lines
 \let\@cite@ofmt\@firstofone
 % avoid brackets around text for \cite:
 \def\@biblabel#1{}
 \def\@cite#1#2{{#1\if@tempswa , #2\fi}}
\makeatother
\newlength{\cslhangindent}
\setlength{\cslhangindent}{1.5em}
\newlength{\csllabelwidth}
\setlength{\csllabelwidth}{3em}
\newenvironment{CSLReferences}[2] % #1 hanging-indent, #2 entry-spacing
 {\begin{list}{}{%
  \setlength{\itemindent}{0pt}
  \setlength{\leftmargin}{0pt}
  \setlength{\parsep}{0pt}
  % turn on hanging indent if param 1 is 1
  \ifodd #1
   \setlength{\leftmargin}{\cslhangindent}
   \setlength{\itemindent}{-1\cslhangindent}
  \fi
  % set entry spacing
  \setlength{\itemsep}{#2\baselineskip}}}
 {\end{list}}
\usepackage{calc}
\newcommand{\CSLBlock}[1]{\hfill\break\parbox[t]{\linewidth}{\strut\ignorespaces#1\strut}}
\newcommand{\CSLLeftMargin}[1]{\parbox[t]{\csllabelwidth}{\strut#1\strut}}
\newcommand{\CSLRightInline}[1]{\parbox[t]{\linewidth - \csllabelwidth}{\strut#1\strut}}
\newcommand{\CSLIndent}[1]{\hspace{\cslhangindent}#1}



\setlength{\emergencystretch}{3em} % prevent overfull lines

\providecommand{\tightlist}{%
  \setlength{\itemsep}{0pt}\setlength{\parskip}{0pt}}



 


\usepackage{longtable}
\usepackage{amsmath}
\usepackage{amssymb}
\usepackage{multicol}
\usepackage{float}
\usepackage{typearea}
\floatplacement{figure}{htbp}
\floatplacement{table}{htbp}
\AtBeginDocument{%
  \storeareas\normalpapersize
}
\BeforeRestoreareas{\cleardoublepage}
\newcommand*\uselandscape{%
  \cleardoublepage
  \KOMAoptions{paper=landscape}%
  \recalctypearea
  \areaset{1.2\textwidth}{1.2\textheight}%
}
\KOMAoption{captions}{tableheading}
\makeatletter
\@ifpackageloaded{caption}{}{\usepackage{caption}}
\AtBeginDocument{%
\ifdefined\contentsname
  \renewcommand*\contentsname{Table of contents}
\else
  \newcommand\contentsname{Table of contents}
\fi
\ifdefined\listfigurename
  \renewcommand*\listfigurename{List of Figures}
\else
  \newcommand\listfigurename{List of Figures}
\fi
\ifdefined\listtablename
  \renewcommand*\listtablename{List of Tables}
\else
  \newcommand\listtablename{List of Tables}
\fi
\ifdefined\figurename
  \renewcommand*\figurename{Figure}
\else
  \newcommand\figurename{Figure}
\fi
\ifdefined\tablename
  \renewcommand*\tablename{Table}
\else
  \newcommand\tablename{Table}
\fi
}
\@ifpackageloaded{float}{}{\usepackage{float}}
\floatstyle{ruled}
\@ifundefined{c@chapter}{\newfloat{codelisting}{h}{lop}}{\newfloat{codelisting}{h}{lop}[chapter]}
\floatname{codelisting}{Listing}
\newcommand*\listoflistings{\listof{codelisting}{List of Listings}}
\makeatother
\makeatletter
\makeatother
\makeatletter
\@ifpackageloaded{caption}{}{\usepackage{caption}}
\@ifpackageloaded{subcaption}{}{\usepackage{subcaption}}
\makeatother
\usepackage{bookmark}
\IfFileExists{xurl.sty}{\usepackage{xurl}}{} % add URL line breaks if available
\urlstyle{same}
\hypersetup{
  pdftitle={An agent-based simulation of psychological resilience},
  pdfauthor={Aly Lamuri},
  colorlinks=true,
  linkcolor={blue},
  filecolor={Maroon},
  citecolor={Blue},
  urlcolor={Blue},
  pdfcreator={LaTeX via pandoc}}


\title{An agent-based simulation of psychological resilience}
\author{Aly Lamuri}
\date{}
\begin{document}
\maketitle


\section{Introduction}\label{introduction}

Anxiety and depression are among the most prevalent psychological
disorders in the general population, generating substantial
socioeconomic consequences (GBD Collaborators 2022; Arias, Saxena, and
Verguet 2022). These conditions impose considerable direct and indirect
costs, while curative and rehabilitative approaches often demonstrate
limited effectiveness (Mekonen et al. 2021). Strengthening psychological
resilience offers a preventive pathway to reduce vulnerability to such
disorders (Troy et al. 2023). Public health policy should therefore
prioritize preventive strategies that foster resilience. However,
designing cost-effective preventive policies requires rigorous
assessment methods. Agent-based modeling (ABM) provides a promising
approach to simulate psychological resilience and generate evidence to
guide interventions aimed at enhancing resilience at the population
level.

\subsection{Previous ABM Publications}\label{previous-abm-publications}

The following five articles were selected for their relevance to the
development of ABM simulations addressing mental health, social
interaction, or their intersection. Each study contributed unique
methodological insights that informed the current work. While some
articles demonstrated only minimal overlap with our research, they were
included because they resolved specific modeling challenges pertinent to
our framework. Collectively, these studies provided a foundation for
constructing comprehensive and realistic ABM simulations in the context
of mental health and social dynamics.

Romanyukha et al. (2023) implemented an ABM to evaluate the mental
health consequences of policy measures during the COVID-19 pandemic. The
model integrated epidemic dynamics with the development of mental
disorders in a large urban population. Using Quality-Adjusted Life Years
(QALY) as an outcome measure, the study considered major depressive
disorder, anxiety disorder, COVID-19 cases (lethal and non-lethal), and
immunization. The results showed that mental disorders accounted for a
substantial share of pandemic-related health losses, and under strong
lockdown conditions, the lowest QALY loss occurred when approximately
70\% of the population was isolated. These findings demonstrate that ABM
can quantify the impact of policy interventions on mental health
outcomes, supporting its use for evaluating preventive strategies that
enhance psychological resilience.

Murase et al. (2021) introduced a generalized weighted social network
(GWSN) model to describe the formation of social ties and networks. The
GWSN model integrates mechanisms previously examined in isolation,
including triadic closure (the tendency for friends of friends to become
friends), homophilic interactions, and link termination processes. Due
to the model's complexity and numerous input parameters, the authors
combined extensive simulations with deep neural networks for regression
and global sensitivity analysis, enabling prediction of network
properties and identification of key determinants of network dynamics.
By providing a robust framework for modeling realistic social
interactions, this approach is directly relevant for simulating how
social connectivity influences resilience and the spread of mental
disorders in ABM.

Andarlia and Gunawan (2021) examined the contagion effect of
psychological disorders using ABM to simulate the spread of depression
in a population. The model accounted for the influence of close social
relationships, which could trigger mild, moderate, or severe depressive
episodes, and modeled recovery through therapy with transition dynamics
and recovery rates differentiated by gender. Simulation results
indicated that higher contact rates with depressed individuals increased
episode severity, whereas greater therapy uptake reduced depression
prevalence. This study highlights how ABM can capture both the
propagation and mitigation of mental disorders through social
mechanisms, a key component for modeling interventions aimed at
strengthening psychological resilience.

López and Giovanini (2024) introduced a methodology embedding an ABM
within adaptive temporal networks to simulate daily interactions and
examine the interplay between infectious disease spread and individual
behavior. The model integrates individual behavior, social dynamics, and
epidemiological factors, validated using real-world influenza outbreak
data, and can simulate complex social phenomena such as social
awareness. By incorporating self-organized system logic, agents respond
dynamically to external stimuli, perceptions, and health states.
Combined with social network models and mental health contagion
dynamics, this approach demonstrates that ABM can simulate emergent
phenomena arising from the interaction of social behavior and health,
providing a powerful tool for testing policies that influence population
resilience.

Benny et al. (2022) employed ABM to evaluate the impact of government
policies on mental health, focusing on depression in expectant mothers.
The model incorporated parameters from the ``All Our Families'' cohort
dataset and literature, including household income, age, education, and
environmental factors. The study hypothesized that income-supportive
programs would reduce maternal depression, and simulations confirmed
that progressive income policies significantly lowered prevalence.
Additionally, modeling mothers' social networks showed that expanding
social connections further reduced depression risk. These findings
illustrate that ABM can assess the effectiveness of policy interventions
on mental health outcomes, underscoring its applicability for designing
strategies to enhance resilience and reduce the prevalence of
psychological disorders.

\subsection{Current Research}\label{current-research}

This study simulates stress perception using an ABM approach. Stress
perception is operationalized as a generalization of the Perceived
Stress Scale-10 (PSS-10), following the guidance provided in its
interpretation manual (Cohen 1988). Modeling stress perception is
essential for capturing the activation of psychological resilience
(Parsons, Kruijt, and Fox 2016). Under equivalent stress exposure,
agents with higher resilience are less likely to develop psychological
disorders, whereas less resilient agents are more vulnerable. The
simulation incorporates multiple psychological resources, including
social support, family support, and psychological capital, with the
latter encompassing personality traits such as openness,
conscientiousness, and extraversion (Hobfoll 1989; Klaver et al. 2021;
Zager Kocjan, Kavčič, and Avsec 2021).

The relevance of ABM lies in its ability to simulate complex social
dynamics that traditional methods cannot capture. Psychological
resilience operates through homeostatic mechanisms influenced by social
interactions, where supportive networks enhance resource conservation
and mitigate stress (Hobfoll 2002; Fasihi Harandi, Mohammad Taghinasab,
and Dehghan Nayeri 2017). ABM allows for the representation of
heterogeneous agents with varying baseline characteristics, enabling the
exploration of how agent interactions affect population-level resilience
(Andarlia and Gunawan 2021; López and Giovanini 2024). Our ABM
implementation aimed to resolve the question:

\begin{quote}
Can agent-based simulations effectively model the homeostatic mechanisms
of psychological resilience by incorporating social network dynamics and
agent interactions that either support or deteriorate mental health
outcomes?
\end{quote}

By configuring the base population in an agent-based simulation, we can
simulate the homeostatic mechanism of psychological resilience, where
agent interactions determine outcomes. Population metrics will remain
within a constant range, reflecting a stable homeostatic state.

\section{Methods}\label{methods}

\subsection{Modelling Approach}\label{modelling-approach}

The core components from prior ABM literature (Romanyukha et al. 2023;
Murase et al. 2021; Andarlia and Gunawan 2021; López and Giovanini 2024;
Benny et al. 2022) were integrated into a unified framework for
evaluating the homeostasis mechanism of the human psyche. Building on
established social network dynamics and mental health contagion
mechanisms, our model incorporated comprehensive stress perception via
the Perceived Stress Scale-10 (PSS-10), challenge-hindrance appraisal,
and resilience dynamics, enabling simulation of homeostatic mechanisms
that determine population-level mental health outcomes.

The simulation was operated through structured daily steps: (1) agent
initialization with baseline resilience, affect, and resources; (2)
stress event generation with controllability and overload attributes;
(3) appraisal yielding challenge/hindrance scores; (4) coping decisions
influenced by social networks; (5) mutual affect and resilience
exchanges; (6) resource allocation across protective factors; and (7)
homeostatic adjustments toward equilibrium. Network adaptation occurred
when chronic stress triggers rewiring based on homophily and support
effectiveness.

Key departures from prior work included PSS-10 integration for validated
stress measurement and explicit resource dynamics with softmax
allocation. Literature-supported elements include social contagion
(Andarlia and Gunawan 2021) and network formation (Murase et al. 2021),
while PSS-10 bifactor modeling and resilience mechanisms remain
empirically grounded but require further validation against longitudinal
data. Model is highly applicable for simulating psychological
resilience, though limitations include unvalidated parameter ranges,
necessitating sensitivity analysis.

\setlength\LTleft{0pt}
\setlength\LTright{0pt}

\begin{longtable}{ll}

\caption{\label{tbl-parameters}Summary of model parameters and variables used in the agent-based mental health simulation}

\tabularnewline

 \\
 \\
\toprule
\multicolumn{2}{c}{\textbf{Model Structure}} \\
\midrule
$N$ & Network size (number of agents) \\
$WS_k$ & Mean degree in Watts-Strogatz network topology \\
$WS_p$ & Rewiring probability in Watts-Strogatz network \\
\midrule
\multicolumn{2}{c}{\textbf{Agent Initialization}} \\
\midrule
$\mu_{\mathfrak{R}, \text{0}}$ & Initial resilience mean \\
$\sigma_{\mathfrak{R},\text{0}}$ & Initial resilience standard deviation \\
$\mu_{A, \text{0}}$ & Initial affect mean \\
$\sigma_{A,\text{0}}$ & Initial affect standard deviation \\
$\mu_{R, \text{0}}$ & Initial resources mean \\
$\sigma_{R,\text{0}}$ & Initial resources standard deviation \\
$X$ & Random variable for baseline generation \\
\midrule
\multicolumn{2}{c}{\textbf{Stress Processing}} \\
\midrule
$\omega_c$ & Controllability weight in appraisal \\
$\omega_o$ & Overload weight in appraisal \\
$b$ & Bias term in appraisal function \\
$\gamma$ & Sigmoid steepness parameter \\
$\eta_{\chi}$ & Challenge threshold modifier \\
$\eta_{\zeta}$ & Hindrance threshold modifier \\
$\eta_{\text{0}}$ & Base stress threshold \\
\midrule
\multicolumn{2}{c}{\textbf{Coping}} \\
\midrule
$p_b$ & Base probability for successful coping \\
$\theta_{\text{cope,}\chi}$ & Challenge bonus for coping \\
$\theta_{\text{cope,}\zeta}$ & Hindrance penalty for coping \\
$\delta_{\text{cope,soc}}$ & Social influence on coping \\
\midrule
\multicolumn{2}{c}{\textbf{Affect Dynamics}} \\
\midrule
$\alpha_p$ & Peer influence rate \\
$\alpha_e$ & Event appraisal rate \\
$\lambda_{\text{affect}}$ & Affect homeostasis rate \\
$k$ & Number of neighbors \\
$k_{\text{influence}}$ & Number of influencing neighbors \\
\midrule
\multicolumn{2}{c}{\textbf{Resilience Dynamics}} \\
\midrule
$\lambda_{\text{resilience}}$ & Resilience homeostasis rate \\
$\theta_{\text{boost}}$ & Resilience boost rate \\
$\theta_{\text{boost|cope}}$ & Boost rate for successful coping \\
$\alpha_s$ & Social support rate for resilience \\
$\eta_{\text{res,overload}}$ & Threshold for overload effects \\
$h_c$ & Consecutive hindrances count \\
$F$ & Set of protective factors \\
$e_f$ & Efficacy of factor $f$ \\
\midrule
\multicolumn{2}{c}{\textbf{Resource}} \\
\midrule
$\lambda_R$ & Resource regeneration rate \\
$\beta_{\text{softmax}}$ & Softmax temperature parameter \\
$\beta_a$ & Affect influence parameter \\
$R_{\max}$ & Maximum resources \\
$R_a$ & Available resources \\
$w_f$ & Allocation weight for factor $f$ \\
$r_f$ & Resources allocated to factor $f$ \\
$\gamma_p$ & Protective improvement rate \\
\midrule
\multicolumn{2}{c}{\textbf{Social Network}} \\
\midrule
$\delta_{\text{homophily}}$ & Homophily strength \\
$\eta_{\text{exchange}}$ & Support exchange threshold \\
$s_{ij}$ & Similarity between agents \\
$e_s$ & Support effectiveness \\
$p_{\text{keep}}$ & Connection retention probability \\
$c_{\text{breach}}$ & Stress breach count \\
$\eta_{\text{adapt}}$ & Adaptation threshold \\
$p_{\text{rewire}}$ & Rewiring probability \\
$A_i, A_j$ & Affect values for agents $i,j$ \\
$\mathfrak{R}_i, \mathfrak{R}_j$ & Resilience values for agents $i,j$ \\
\midrule
\multicolumn{2}{c}{\textbf{PSS-10}} \\
\midrule
$\mu_c$ & Controllability dimension mean \\
$\sigma_c$ & Controllability dimension standard deviation \\
$\mu_o$ & Overload dimension mean \\
$\sigma_o$ & Overload dimension standard deviation \\
$\rho_\Psi$ & PSS-10 dimension correlation \\
$\eta_\Psi$ & PSS-10 stress threshold \\
$\Psi_i$ & PSS-10 item response for item $i$ \\
$\lambda_{c,\Psi,i}$ & Factor loading for item $j$ on controllability dimension \\
$\lambda_{o,\Psi,i}$ & Factor loading for item $j$ on overload dimension \\
$\epsilon$ & Measurement error \\
$c_\Psi$ & PSS-10 controllability dimension \\
$o_\Psi$ & PSS-10 overload dimension \\
$\Psi$ & Total PSS-10 score \\
\midrule
\multicolumn{2}{c}{\textbf{Core Variables}} \\
\midrule
$\mathfrak{R}$ & Agent resilience level \\
$A$ & Agent affect level \\
$R$ & Agent resource level \\
$S$ & Agent stress level \\
$z$ & Weighted appraisal score \\
$\chi$ & Challenge component \\
$\zeta$ & Hindrance component \\
$\eta_{\mathrm{eff}}$ & Effective stress threshold \\
$p_{\mathrm{coping}}$ & Probability of successful coping \\
$c$ & Controllability \\
$o$ & Overload \\
$s$ & Event magnitude \\
$\delta$ & Polarity effect strength \\
$L$ & Appraised stress load \\
$A_t$ & Current affect at time t \\
$\mathfrak{R}_t$ & Current resilience at time t \\
$S_t$ & Current stress at time t \\
$\mathfrak{R}_0$ & Baseline resilience \\
\midrule
\multicolumn{2}{c}{\textbf{Daily Integration Variables}} \\
\midrule
$\bar{\chi}_d$ & Daily average challenge \\
$\bar{\zeta}_d$ & Daily average hindrance \\
$\Delta A_p$ & Peer influence effect on affect \\
$\Delta A_e$ & Event appraisal effect on affect \\
$\Delta A_h$ & Homeostatic effect on affect \\
$\Delta \mathfrak{R}_p$ & Protective factor contribution \\
$\delta_{\text{stress}}$ & Stress decay rate \\
$\lambda_s$ & Subevent rate parameter \\
$\mathcal{P}(\lambda)$ & Poisson distribution \\
$\Delta \mathfrak{R}_{\chi\zeta}$ & Resilience change from challenge/hindrance \\
$\Delta \mathfrak{R}_p$ & Resilience boost from protective factors \\
$\Delta \mathfrak{R}_o$ & Resilience change from overload \\
$\Delta \mathfrak{R}_s$ & Resilience change from social support \\
$\Delta A_{t+1}$ & Affect change at time t+1 \\
$\mathfrak{R}_{t+1}$ & Resilience at time t+1 \\
$S_{t+1}$ & Stress level at time t+1 \\
$\mathrm{clamp}(x, a, b)$ & Clamping function to bind the value of $x$ into the range of $[a, b]$ \\
$n_s$ & Number of subevents per day \\
$n_e$ & Number of stress events in day \\
\bottomrule

\end{longtable}

\subsection{Simulation Mechanism}\label{simulation-mechanism}

\subsubsection{Agent Initialization and State
Management}\label{agent-initialization-and-state-management}

Each agent was initialized with baseline values representing natural
equilibrium points using mathematical transformations ensuring proper
statistical distributions. Resilience baselines utilized sigmoid
transformation and affect baselines employed hyperbolic tangent
transformation, both using seeded random number generation for
reproducibility:

\begin{align*}
\mathfrak{R}_{\text{0}} &= \sigma\left(\frac{X - \mu_{\mathfrak{R},\text{0}}}{\sigma_{\mathfrak{R},\text{0}}}\right) \\
R_{\text{0}} &= \sigma\left(\frac{X - \mu_{R,\text{0}}}{\sigma_{R,\text{0}}}\right) \\
A_{\text{0}} &= \tanh\left(\frac{X - \mu_{A,\text{0}}}{\sigma_{A,\text{0}}}\right)
\end{align*} \label{eq-baseline-transformations}

where \(X \sim \mathcal{N}(\mu, \sigma^2)\) is a normal random variable,
\(\sigma(x) = \frac{1}{1+e^{-x}}\) is the sigmoid function, and
\(\tanh(x) = \frac{e^x - e^{-x}}{e^x + e^{-x}}\) is the hyperbolic
tangent function.

Core state variables included resilience (\(\mathfrak{R} \in [0,1]\)),
affect (\(A \in [-1,1]\)), resources (\(R \in [0,1]\)), and current
stress (\(S \in [0,1]\)). The model integrated comprehensive Perceived
Stress Scale-10 (PSS-10) functionality using a bifactor model with
dimension score generation and validated theoretical correlations:

\begin{equation}\phantomsection\label{eq-pss10-dimension-score}{
\begin{pmatrix}
c_\Psi \\ o_\Psi
\end{pmatrix}
\sim
\mathcal{N}\left(
\begin{pmatrix}
\mu_c \\ \mu_o
\end{pmatrix},
\begin{pmatrix}
\sigma_c^2 & \rho_\Psi \sigma_c \sigma_o \\
\rho_\Psi \sigma_c \sigma_o & \sigma_o^2
\end{pmatrix}
\right)
}\end{equation}

Where \(c_\Psi, o_\Psi \in [0,1]\) are PSS-10 dimension scores,
\(\rho_\Psi \in [-1,1]\) is the bifactor correlation, and
\(\mathcal{N}(\boldsymbol{\mu}, \boldsymbol{\Sigma})\) denotes the
multivariate normal distribution.

\subsubsection{Stress Event Processing and
Coping}\label{stress-event-processing-and-coping}

The stress processing pipeline transformed life events into
psychological responses through challenge-hindrance appraisal. Events
were generated with controllability (\(c\)) and overload (\(o\))
attributes, then appraised using the following mathematical framework:

\begin{align}
z &= \omega_c \cdot c - \omega_o \cdot o + b \\
\chi &= \sigma(\gamma \cdot z) \\
\zeta &= 1 - \chi \\
\eta_{\mathrm{eff}} &= \eta_{\text{0}} + \eta_{\chi} \cdot \chi - \eta_{\zeta} \cdot \zeta \\
p_{\mathrm{coping}} &= p_b + \theta_{\text{cope,}\chi} \cdot \chi - \theta_{\text{cope,}\zeta} \cdot \zeta + \delta_{\text{cope,soc}} \cdot \frac{1}{k} \sum_{j=1}^k A_j
\label{eq-stress-processing}
\end{align}

where \(c, o \in [0,1]\) are event attributes,
\(\omega_c, \omega_o \in \mathbb{R}\) are appraisal weights,
\(b \in \mathbb{R}\) is bias, \(\gamma > 0\) controls sigmoid steepness,
\(\eta_{\text{0}}, \eta_{\chi}, \eta_{\zeta} \in [0,1]\) are threshold
parameters, \(p_b \in [0,1]\) is base coping probability,
\(\theta_{\text{cope,}\chi}, \theta_{\text{cope,}\zeta} > 0\) are coping
modifiers, \(\delta_{\text{cope,soc}} \in [0,1]\) is social influence,
\(k\) is number of neighbors, and \(A_j \in [-1,1]\) are neighbor affect
values.

Dynamic threshold evaluation adjusted stress response based on event
characteristics, and coping success determination integrated
challenge/hindrance effects with social influence through probability
calculation.

\subsubsection{Social Interaction
Mechanism}\label{social-interaction-mechanism}

Social interactions occurred between neighboring agents in the network
topology, enabling emotional contagion and mutual support. The system
processed interactions through utility functions computing mutual
emotional and resilience effects, with social influence on affect
determined by neighbor states and relationship dynamics.

\paragraph{Mutual Influence Mechanism}\label{mutual-influence-mechanism}

Social interactions created bidirectional changes where both individuals
affect each other's emotional state. The model recognized that negative
emotional states tended to have stronger influence than positive ones:

\begin{equation}\phantomsection\label{eq-mutual-influence}{
\Delta A_i = \alpha_p \cdot (A_j - A_i) \cdot \begin{cases}
1.5 & \text{if } A_j - A_i < 0 \\
1.0 & \text{if } A_j - A_i \geq 0
\end{cases}
}\end{equation}

Where \(\Delta A_i\) is affect change for agent \(i\),
\(\alpha_p \in [0,1]\) is peer influence rate, and
\(A_i, A_j \in [-1,1]\) are affect values.

\paragraph{Network Adaptation
Mechanism}\label{network-adaptation-mechanism}

When individuals experienced repeated stress, they adapted their social
connections to better suit their needs. Network adaptation triggers
based on stress breach counts:

\begin{equation}\phantomsection\label{eq-adaptation-trigger}{
\mathrm{trigger\ adaptation} = \begin{cases}
1 & \text{if } c_{\text{breach}} \geq \eta_{\text{adapt}} \\
0 & \text{otherwise}
\end{cases}
}\end{equation}

Where \(c_{\text{breach}} \in \mathbb{N}\) is stress breach count and
\(\eta_{\text{adapt}} \in \mathbb{N}\) is adaptation threshold.
\(c_{\text{breach}}\) is a mechanism to record chronic stress pattern,
reflecting the count of perceived stress exceeding
\(\eta_{\mathrm{eff}}\).

Connection preferences were determined by similarity and support
effectiveness:

\begin{equation}\phantomsection\label{eq-connection-similarity}{
s_{ij} = 1 - \frac{|A_i - A_j| + |\mathfrak{R}_i - \mathfrak{R}_j|}{2}
}\end{equation}

Where \(s_{ij} \in [0,1]\) is similarity between agents \(i,j\),
\(A_i, A_j \in [-1,1]\) are affect values, and
\(\mathfrak{R}_i, \mathfrak{R}_j \in [0,1]\) are resilience values.

Connection retention probability balanced homophily with support
effectiveness:

\begin{equation}\phantomsection\label{eq-retention-probability}{
p_{\text{keep}} = s_{ij} \cdot \delta_{\text{homophily}} + e_s \cdot (1 - \delta_{\text{homophily}})
}\end{equation}

where \(p_{\text{keep}} \in [0,1]\) is probability of keeping
connection, \(\delta_{\text{homophily}} \in [0,1]\) is homophily
strength, and \(e_s \in [0,1]\) is support effectiveness.

\paragraph{Social Support Dynamics}\label{social-support-dynamics}

Support effectiveness depended on the neighbor's resilience and affect:

\begin{equation}\phantomsection\label{eq-support-effectiveness}{
e_s = \frac{\mathfrak{R}_j + (1 + A_j)/2}{2} + 0.2
}\end{equation}

where \(e_s \in [0,1]\) is support effectiveness,
\(\mathfrak{R}_j \in [0,1]\) is neighbor's resilience, and
\(A_j \in [-1,1]\) is neighbor's affect.

Social support exchange was detected when meaningful improvements
occurred:

\begin{equation}\phantomsection\label{eq-support-exchange}{
\mathrm{support\ exchange} = \begin{cases}
1 & \text{if } \max\{|\Delta A_i|, |\Delta \mathfrak{R}_i|, |\Delta A_j|, |\Delta \mathfrak{R}_j|, |\Delta \text{resources}|\} > \eta_{\text{exchange}} \\
0 & \text{otherwise}
\end{cases}
}\end{equation}

where \(\Delta A_i, \Delta A_j\) are affect changes,
\(\Delta \mathfrak{R}_i, \Delta \mathfrak{R}_j\) are resilience changes,
\(\Delta \text{resources}\) is resource transfer, and
\(\eta_{\text{exchange}} \in [0,1]\) is exchange threshold.

\subsubsection{Daily Simulation Step}\label{daily-simulation-step}

Each simulation day followed a structured sequence ensuring proper
mechanism integration. The process initiated with daily initialization,
capturing initial affect and resilience values while obtaining neighbor
emotional states. Subevent generation determined daily activity through
Poisson sampling, creating random sequences of interactions and stress
events:

\begin{align}
n_s &\sim \max(\mathcal{P}(\lambda_s), 1) \\
\bar{\chi}_d &= \frac{1}{n_e}\sum_{i=1}^{n_e} \chi_i \\
\bar{\zeta}_d &= \frac{1}{n_e}\sum_{i=1}^{n_e} \zeta_i
\label{eq-daily-integration}
\end{align}

where \(n_s \in \mathbb{N}\) is number of subevents;
\(\mathcal{P}(\lambda_s)\) is Poisson distribution with rate
\(\lambda_s\); \(\bar{\chi}_d, \bar{\zeta}_d \in [0,1]\) are daily
averages; and \(n_e\) is number of stress events.

Daily integration normalized challenge/hindrance values by event count,
providing inputs for dynamics application. Dynamics updates applied
integrated affect and resilience dynamics:

\begin{align}
A_{t+1} &= A_t + \Delta A_p + \Delta A_e + \Delta A_h \\
\mathfrak{R}_{t+1} &= \mathfrak{R}_t + \Delta \mathfrak{R}_{\chi\zeta} + \Delta \mathfrak{R}_p + \Delta \mathfrak{R}_o + \Delta \mathfrak{R}_s + \lambda_{\text{resilience}} \cdot (\mathfrak{R}_{\text{0}} - \mathfrak{R}_t) \\
S_{t+1} &= S_t \cdot (1 - \delta_{\text{stress}})
\label{eq-dynamics-updates}
\end{align}

where \(A_t, \mathfrak{R}_t, S_t \in [0,1]\) are current values;
\(\Delta A_p, \Delta A_e, \Delta A_h\) are affect changes;
\(\Delta \mathfrak{R}_{\chi\zeta}, \Delta \mathfrak{R}_p, \Delta \mathfrak{R}_o, \Delta \mathfrak{R}_s\)
are resilience changes; \(\lambda_{\text{resilience}} \in [0,1]\) is
homeostatic rate; \(\mathfrak{R}_{\text{0}} \in [0,1]\) is baseline
resilience; and \(\delta_{\text{stress}} \in [0,1]\) is stress decay
rate.

Homeostatic adjustment applied natural pull toward baseline equilibrium
for affect and resilience, while stress decay followed exponential
reduction. Daily reset procedures cleared tracking variables and stored
summaries for analysis.

\subsubsection{Affect and Resilience
Dynamics}\label{affect-and-resilience-dynamics}

Integrated affect dynamics combined peer influence, event appraisal
effects, and homeostasis through the following mathematical framework:

\begin{align}
\Delta A_p &= \frac{1}{k} \sum_{j=1}^{k} \alpha_p \cdot (A_j - A_t) \cdot \mathbb{1}_{j \leq k_{\text{influence}}} \\
\Delta A_e &= \alpha_e \cdot \bar{\chi}_d \cdot (1 - A_t) - \alpha_e \cdot \bar{\zeta}_d \cdot \max(0.1, A_t + 1) \\
\Delta A_h &= \lambda_{\text{affect}} \cdot (A_{\text{0}} - A_t)
\label{eq-affect-dynamics}
\end{align}

where \(\Delta A_p, \Delta A_e, \Delta A_h\) are affect change
components; \(k\) is number of neighbors; \(k_{\text{influence}}\) is
number of influencing neighbors; \(\alpha_p, \alpha_e \in [0,1]\) are
influence rates; \(\bar{\chi}_d, \bar{\zeta}_d \in [0,1]\) are daily
averages; \(A_t, A_j \in [-1,1]\) are affect values;
\(\lambda_{\text{affect}} \in [0,1]\) is homeostatic rate;
\(A_{\text{0}} \in [-1,1]\) is baseline affect; and \(\mathbb{1}\) is
indicator function.

Resilience dynamics integrated challenge-hindrance effects, protective
factor boosts, overload effects, and social support contributions:

\begin{equation}\phantomsection\label{eq-resilience-boosts}{
\Delta \mathfrak{R}_p = \sum_{f \in F} e_f \cdot (\mathfrak{R}_{\text{0}} - \mathfrak{R}_t) \cdot \theta_{\text{boost}}
}\end{equation}

where \(\Delta \mathfrak{R}_p\) is resilience boost from protective
factors;
\(F = \{\mathrm{soc}, \mathrm{fam}, \mathrm{int}, \mathrm{cap}\}\) is
set of protective factors; \(e_f \in [0,1]\) is efficacy of factor
\(f\); \(\mathfrak{R}_{\text{0}}, \mathfrak{R}_t \in [0,1]\) are
baseline and current resilience; and \(\theta_{\text{boost}} > 0\) is
boost rate parameter.

Challenge-hindrance effects varied based on coping outcomes: successful
coping yields positive resilience changes while failed coping produces
negative impacts.

\subsubsection{Resource Management
System}\label{resource-management-system}

Resources represented finite psychological and physical capacity for
coping and protective factor maintenance. Resource regeneration followed
affect-modulated recovery, while consumption occurred during coping
attempts. Protective factor allocation utilized softmax decision
framework for bounded rational resource distribution across social
support, family support, formal intervention, and psychological capital:

\begin{align}
R' &= \lambda_R \cdot (R_{\max} - R) \cdot (1 + \beta_a \cdot \max(0, A)) \\
w_f &= \frac{\exp(e_f / \beta_{\text{softmax}})}{\sum_{k \in F} \exp(e_k / \beta_{\text{softmax}})}
\label{eq-resource-dynamics}
\end{align}

where \(R' > 0\) is resource regeneration; \(\lambda_R \in [0,1]\) is
regeneration rate; \(R_{\max} = 1\) is maximum resources;
\(R \in [0,1]\) is current resources; \(\beta_a > 0\) is affect
influence parameter; \(A \in [-1,1]\) is current affect;
\(w_f \in [0,1]\) is allocation weight for factor \(f\);
\(e_f \in [0,1]\) is efficacy of factor \(f\);
\(\beta_{\text{softmax}} > 0\) is softmax temperature; and \(F\) is set
of protective factors.

\subsubsection{Model-Level Simulation
Orchestration}\label{model-level-simulation-orchestration}

The simulation employed Mesa's agent-based modeling framework with
dual-class architecture separating agent behaviors from model
orchestration. Network structure utilized Watts-Strogatz small-world
topology providing realistic social connection patterns with local
clustering and short path lengths. The network was initialized with a
mean degree of \(WS_k\) and rewiring probability of \(WS_p\).

Model-level coordination managed population statistics, network
adaptation tracking, and cumulative social support monitoring. The
orchestration ensured proper temporal sequencing and data collection
while maintaining computational efficiency for large-scale simulations.

\subsubsection{Data Collection and Analysis
Framework}\label{data-collection-and-analysis-framework}

The simulation implemented Mesa's \texttt{DataCollector} for
standardized, efficient data collection replacing manual tracking
systems. Model-level reporters captured population metrics including
average PSS-10 scores (\(\bar{\Psi}\)), resilience levels
(\(\bar{\mathfrak{R}}\)), affect states (\(\bar{A}\)), stress prevalence
(\(P_{\text{stressed}}\)), and social network characteristics.
Agent-level reporters tracked individual trajectories for PSS-10 scores,
resilience, affect, resources, current stress, stress controllability,
stress overload, and consecutive hindrances. Recovery potential,
vulnerability index, challenge-hindrance balance, and coping success
rates were also computed for comprehensive analysis. The data collection
framework supported comprehensive research applications including
baseline versus intervention comparison, individual trajectory analysis,
network analysis, and parameter sensitivity studies.

\subsection{Data Analysis}\label{data-analysis}

Exploratory data analysis was conducted on the exported simulation data
at both the agent level and the population level. Agent-level data
encompassed individual data for each simulation epoch, capturing the
unique trajectories of each agent over time. In contrast,
population-level data consisted of aggregated metrics for each
simulation epoch, providing a summary view of collective behavior across
the entire agent population.

For agent-level data, the first and final epochs were extracted to
examine initial and end-state distributions. Histograms were visualized
for the core variables: resilience, affect, stress, and PSS-10, enabling
assessment of shifts in individual agent characteristics throughout the
simulation. For population-level data, stability over time was assessed
by computing moving averages to smooth temporal fluctuations and
highlight long-term trends. Specifically, 7-step moving averages (7-MA),
14-step moving averages (14-MA), and 28-step moving averages (28-MA)
were calculated for key metrics, allowing evaluation of whether
population-level indicators remained within stable ranges or exhibited
significant deviations.

The rationale for this analytical approach stems from the research
question and hypotheses outlined in the study. The primary question
investigates whether agent-based simulations can effectively model the
homeostatic mechanisms of psychological resilience through social
network dynamics and agent interactions. By analyzing agent-level data,
we examined how individual resilience, affect, stress, and PSS-10 scores
evolved, reflecting the activation of psychological resilience under
stress exposure. The visualization of histograms for initial and final
epochs allowed for the identification of heterogeneity in agent
responses, supporting the hypothesis that agents with higher resilience
are less likely to develop psychological disorders.

At the population level, the assessment of stability through moving
averages tested the hypothesis that population metrics would remain
within a constant range, indicative of a stable homeostatic state. This
approach captured the influence of social interactions and resource
dynamics on collective mental health outcomes, enabling validation of
whether the simulation accurately represented homeostatic mechanisms.
Together, these analyses provided empirical evidence for the model's
ability to simulate complex social dynamics that traditional methods
cannot capture, thereby addressing the core inquiry into the efficacy of
agent-based modeling for understanding psychological resilience.

\section{Results}\label{results}

Figure~\ref{fig-initial-population} presents a cross-sectional snapshot
of agent psychological states at epoch 0, derived from randomly
generated baseline data with minimal theoretical associations except for
PSS-10 scores and stress levels. This figure underscores the absence of
predetermined relationships in the initialization phase, highlighting
the random nature of agent attributes that establish a neutral starting
point for subsequent dynamics. The initial stress level has an
artificially strong correlation to PSS-10 because it was generated from
PSS-10. The rationale is to ground the initial stress level to an
empirical values reflected in PSS-10. By illustrating uncorrelated
variables, it emphasizes the model's reliance on stochastic processes to
simulate real-world heterogeneity without biasing initial conditions.

\begin{figure}

\centering{

\pandocbounded{\includegraphics[keepaspectratio]{figures/base_initial_population.pdf}}

}

\caption{\label{fig-initial-population}Cross-sectional visualization of
agent psychological states at epoch 0, highlighting randomly generated
baseline distributions with minimal associations except between PSS-10
and stress levels}

\end{figure}%

Figure~\ref{fig-final-population} captures cross-sectional agent states
at the simulation's conclusion, revealing theoretically grounded
associations among psychological variables. Resilience exhibits a weak
positive correlation with affect, a weak-to-moderate negative
correlation with stress, and a weak negative correlation with PSS-10. In
contrast, stress shows a moderate positive correlation with PSS-10,
accompanied by increasing dispersion in stress levels at higher PSS-10
scores, reflecting realistic overdispersion in psychometric instruments.
This visualization demonstrates model convergence toward expected
relationships, validating the simulation's capacity to generate
plausible homeostatic mechanism of psychological dynamics.

Figure~\ref{fig-time-series} depicts longitudinal trajectories of all
agents across simulation epochs, showcasing stable temporal variation
indicative of effective homeostatic mechanisms. Drawing from
comprehensive agent-state data over time, this figure reveals
equilibrium maintenance through consistent fluctuations, with
implications for understanding behavioral stability and dynamic
resilience in response to stressors. It directly addresses the
hypothesis by evidencing stable population metrics within constant
ranges, affirming that agent-based simulations can model homeostatic
resilience via social network dynamics, where interactions sustain
mental health equilibrium without excessive variability.

\begin{figure}

\centering{

\pandocbounded{\includegraphics[keepaspectratio]{figures/base_final_population.pdf}}

}

\caption{\label{fig-final-population}The emerging associations of agent
psychological states captured at simulation completion}

\end{figure}%

Under homogeneous initial conditions simulating a general population,
the model exhibited stable homeostatic mechanisms with population-level
metrics confined to narrow ranges, directly supporting the research
question on agent-based modeling of psychological resilience through
social network dynamics. Model-level analysis reveals remarkable
stability in aggregate measures. Mean perceived stress (PSS-10 = 24.804
± 0.308, range: 21.93--25.65) demonstrates low variability indicative of
effective homeostatic regulation. Resilience (0.523 ± 0.008, range:
0.48--0.54) shows uniform coping capacity across the population. Affect
(−0.024 ± 0.003, range: −0.03 to −0.01) exhibits controlled mood
variation near neutral. This population-level stability occurs despite
substantial individual-level heterogeneity. Agent-level variability in
stress shows a coefficient of variation of 31.8\%. Resilience
demonstrates a coefficient of variation of 55.3\%. Affect exhibits a
coefficient of variation of 1421\%. Elevated resource levels (0.702 ±
0.005) indicate effective resource management through social network
dynamics. Moderate stress (0.484 ± 0.007) relative to baselines suggests
effective stress mitigation. These outcomes directly affirm the
hypothesis that agent interactions determine mental health outcomes
within a stable homeostatic state. The findings simultaneously address
the research question by demonstrating that complex social interactions
can maintain population equilibrium despite individual variability.

\begin{table}

\caption{\label{tbl-model-level}Model-level population statistics demonstrating homeostatic stability}

\centering{

\centering


\begin{tabular}{llp{5.5cm}}
\toprule
\textbf{Variable} & \textbf{Mean $\pm$ SD} & \textbf{Interpretation} \\
\midrule
PSS-10 across agents & 24.804 $\pm$ 0.308 & Stable stress perception with moderate elevation \\
Resilience level & 0.523 $\pm$ 0.008 & Uniform coping capacity across population \\
Affect (valence) & $-0.024 \pm 0.003$ & Controlled mood variation near neutral \\
Coping success rate & 0.448 $\pm$ 0.012 & Reliable coping outcomes maintaining homeostasis \\
Psychological resources & 0.702 $\pm$ 0.005 & Stable resource levels supporting resilience \\
Current stress level & 0.484 $\pm$ 0.007 & Variable patterns within bounded ranges \\
Challenge appraisal & 0.500 $\pm$ 0.007 & Balanced appraisal pattern \\
Hindrance appraisal & 0.500 $\pm$ 0.007 & Balanced appraisal pattern \\
Challenge--hindrance diff. & $-0.000 \pm 0.014$ & Centered near zero, indicating equilibrium \\
Hindrance sequence length & 1.862 $\pm$ 0.066 & Occasional stress streaks but stable overall \\
Coping events per agent & 684.201 $\pm$ 25.642 & Uniform behavioral patterns \\
Social exchanges per agent & 1522.693 $\pm$ 38.794 & Consistent network engagement \\
Support transactions & 395.853 $\pm$ 22.364 & Consistent mutual aid \\
\bottomrule
\end{tabular}

}

\end{table}%

Table~\ref{tbl-model-level} substantiates a homeostatic system sustained
by social network dynamics and agent interactions. The aggregate metrics
demonstrate stability across all psychological variables, with
coefficients of variation ranging from 1.1\% (resource levels) to 5.7\%
(current stress), indicating tight population-level regulation. These
constrained variabilities evidence steady-state dynamics wherein
supportive networks bolster resource preservation and stress
alleviation, supporting the hypothesis of population metrics within
constant ranges. Balanced challenge-hindrance appraisals (0.500 ± 0.007
each, with difference centered at -0.000 ± 0.014) imply neutral
cognitive processing and effective appraisal mechanisms. Uniform social
exchanges (1522.693 ± 38.794 per agent) denote robust network-based
mutual aid that maintains collective stability. Support transactions
(395.853 ± 22.364) further validate the mutual aid system. The
consistent coping success rate (0.448 ± 0.012) across all agents
validates the model's capacity to simulate emergent equilibrium from
individual interactions that promote mental health outcomes.

\begin{figure}

\centering{

\pandocbounded{\includegraphics[keepaspectratio]{figures/base_time_series.pdf}}

}

\caption{\label{fig-time-series}Stable temporal variation and
homeostatic equilibrium of agent psychological states across all
simulation epochs}

\end{figure}%

\begin{table}

\caption{\label{tbl-agent-level}Agent-level temporal dynamics demonstrating individual heterogeneity within population stability}

\centering{

\centering


\begin{tabular}{llp{5.5cm}}
\toprule
\textbf{Variable} & \textbf{Mean $\pm$ SD} & \textbf{Interpretation} \\
\midrule
PSS-10 per agent-step & 24.804 $\pm$ 7.900 & Variable stress levels with realistic range \\
Resilience level & 0.523 $\pm$ 0.289 & Diverse adaptive capacities \\
Emotional state & $-0.024 \pm$ 0.341 & Variable patterns around neutral \\
Psychological resources & 0.702 $\pm$ 0.089 & Fluctuating resource capacity \\
Stress intensity & 0.484 $\pm$ 0.200 & Variable patterns within bounds \\
Controllability perception & 0.477 $\pm$ 0.131 & Diverse agency experiences \\
Overload perception & 0.612 $\pm$ 0.144 & Diverse capacity challenges \\
Hindrance streak length & 0.932 $\pm$ 1.320 & Variable patterns observed \\
\bottomrule
\end{tabular}

}

\end{table}%

Table~\ref{tbl-agent-level} reveals substantial dynamic variability
within the stable population framework, directly supporting the research
question on homeostatic resilience modeling through social network
dynamics. The contrast between population-level stability (coefficient
of variation \textless{} 6\%) and individual-level heterogeneity
(coefficient of variation 31.8-1421\%) exemplifies the dual nature of
the modeled system. Individual agents experience significant
fluctuations in stress (0.00--40.00), resilience (0.00--1.00), and
affect (−1.00--1.00). These variations are constrained by social
interactions that maintain population equilibrium. The comprehensive
range of controllability (0.477 ± 0.131) captures the cognitive
appraisal diversity that drives individual adaptation strategies.
Overload perceptions (0.612 ± 0.144) further demonstrate diverse
capacity challenges. The bounded hindrance streaks (maximum 17.65)
demonstrate the system's capacity to prevent pathological cascades
through network support. These observations directly validate the
hypothesis that agent interactions shape outcomes within a homeostatic
state. The model portrays a complex adaptive system where individual
diversities reflect realistic psychological heterogeneity. Individual
diversities are harmonized through social dynamics to achieve collective
stability. The model thus successfully demonstrates that agent-based
simulations can effectively capture the homeostatic mechanisms of
psychological resilience. Social network dynamics transform individual
variability into population-level equilibrium.

\section{Discussion}\label{discussion}

The simulation results reproduced homeostatic psychological dynamics
through agent-based modeling of social network interactions.
Population-level metrics maintained stable equilibrium despite
individual heterogeneity. This supports the hypothesis that agent
interactions determine mental health outcomes within constrained ranges.

The observed homeostatic stability aligns with resilience research. This
emphasizes social support systems as protective factors against
psychological distress (Hobfoll 1989). The model successfully reproduced
empirically grounded associations between psychological variables.
Resilience showed negative correlations with stress and PSS-10 scores.
These patterns mirror community resilience studies demonstrating
supportive networks enhance resource conservation and stress mitigation
(Fasihi Harandi, Mohammad Taghinasab, and Dehghan Nayeri 2017).

As demonstrated in Figure~\ref{fig-final-population}, the agent
population exhibits bimodal distribution on affect and stress levels.
This pattern appeared as distinct clustering around stable equilibrium
points. It captured complex adaptive system behaviors where local
heterogeneity supports global stability (Yan, Martinez, and Liu 2017).
Previous ABM mental health research demonstrated similar emergent
phenomena through social contagion mechanisms (Andarlia and Gunawan
2021). The current model extends these findings by incorporating
comprehensive appraisal processes and resource management systems. The
simulation framework provides a robust foundation for evaluating
preventive mental health interventions.

The phenomenon emerged from the interaction of multiple mechanisms
operating at different scales. The network adaptation equation
demonstrated how homophily strength balances similarity preferences with
support effectiveness. This maintains network cohesion while enabling
adaptive rewiring. Connection similarity quantified how affect and
resilience differences influence relationship formation patterns.
Network rewiring triggers occurred when chronic stress breaches exceed
adaptation thresholds. Persistently distressed agents modify their
social environments to improve support access. The support effectiveness
equation demonstrated how neighbor resilience and affect combine to
determine mutual aid capacity. These mechanisms collectively produce the
bimodal affect-stress distribution. As observed in
Figure~\ref{fig-final-population}, agents cluster into stable social
groups that maintain collective equilibrium.

The stress processing pipeline transformed life events into
psychological responses through challenge-hindrance appraisal
mechanisms. The appraisal equation combined with sigmoid transformation
produced balanced challenge-hindrance patterns that maintain cognitive
processing neutrality. The effective threshold equation demonstrates how
challenge and hindrance appraisals modify stress response thresholds
dynamically. Coping success probability integrates social influence
through neighbor affect averaging. This produces consistent coping rates
that sustain population-level resilience. Resource management equations
demonstrate affect-modulated recovery. This maintains stable resource
levels supporting sustained coping capacity. The daily integration
framework normalizes challenge-hindrance values across
Poisson-distributed subevents. This ensures consistent temporal dynamics
regardless of daily stress exposure variability. Homeostatic adjustments
provide natural system restoration toward baseline equilibrium. This
creates the stable population metrics observed across all simulation
epochs.

The simulation demonstrated that population-level mental health outcomes
emerge from complex interactions. These occur between individual
characteristics, social network structures, and environmental stressors.
The model validated agent-based approaches for evaluating preventive
strategies that enhance psychological resilience through social support
optimization (Benny et al. 2022). However, this study exclusive focused
on resilience mechanisms without incorporating mental health outcomes.
The validation was primarily through pattern matching rather than
longitudinal empirical integration. The simulation also makes
assumptions of stationary social network structures that ignore
demographic transitions over extended periods. Notwithstanding these
limitations, future research may find practical uses of the simulation
framework. This model can be modified to evaluate the impact of public
mental health policies that leverage social interaction mechanisms.

\section{Conclusion}\label{conclusion}

This study demonstrates that agent-based simulations can model
homeostatic mechanisms of psychological resilience through social
network dynamics and interactive agent behaviors. The mathematical
framework reproduced empirically validated associations among
psychological variables while maintaining population-level stability
through individual heterogeneity. The bimodal affect-stress distribution
emerges naturally from network rewiring and homophily mechanisms,
providing insight into collective mental health patterns. The
integration of stress perception modeling with comprehensive assessment
tools provides new capabilities for evaluating mental health
intervention effectiveness at population scales. The simulation
framework enables systematic exploration of how individual
characteristics, social connections, and environmental factors combine
to determine collective mental health outcomes.

\section*{References}\label{references}
\addcontentsline{toc}{section}{References}

\multicols{2}
\footnotesize

\phantomsection\label{refs}
\begin{CSLReferences}{1}{0}
\bibitem[\citeproctext]{ref-Andarlia2021}
Andarlia, H T, and I Gunawan. 2021. {``An Agent-Based Model of Contagion
Effects in Affected Depression and Its Recovery Process.''}
\emph{Journal of Physics: Conference Series} 1751 (1): 012007.
\url{https://doi.org/10.1088/1742-6596/1751/1/012007}.

\bibitem[\citeproctext]{ref-arias2022quantifying}
Arias, Daniel, Shekhar Saxena, and Stéphane Verguet. 2022.
{``Quantifying the Global Burden of Mental Disorders and Their Economic
Value.''} \emph{EClinicalMedicine} 54.

\bibitem[\citeproctext]{ref-Benny2022}
Benny, Claire, Shelby Yamamoto, Sheila McDonald, Radha Chari, and Roman
Pabayo. 2022. {``Modelling Maternal Depression: An Agent-Based Model to
Examine the Complex Relationship Between Relative Income and
Depression.''} \emph{International Journal of Environmental Research and
Public Health} 19 (7): 4208.
\url{https://doi.org/10.3390/ijerph19074208}.

\bibitem[\citeproctext]{ref-cohen1988perceived}
Cohen, Sheldon. 1988. {``Perceived Stress in a Probability Sample of the
United States.''}

\bibitem[\citeproctext]{ref-FasihiHarandi2017}
Fasihi Harandi, Tayebeh, Maryam Mohammad Taghinasab, and Tayebeh Dehghan
Nayeri. 2017. {``The Correlation of Social Support with Mental Health: A
Meta-Analysis.''} \emph{Electronic Physician} 9 (9): 5212--22.
\url{https://doi.org/10.19082/5212}.

\bibitem[\citeproctext]{ref-gbd2022global}
GBD Collaborators. 2022. {``Global, Regional, and National Burden of 12
Mental Disorders in 204 Countries and Territories, 1990--2019: A
Systematic Analysis for the Global Burden of Disease Study 2019.''}
\emph{The Lancet Psychiatry} 9 (2): 137--50.

\bibitem[\citeproctext]{ref-Hobfoll1989}
Hobfoll, Stevan E. 1989. {``Conservation of Resources: A New Attempt at
Conceptualizing Stress.''} \emph{American Psychologist} 44 (3): 513--24.
\url{https://doi.org/10.1037/0003-066x.44.3.513}.

\bibitem[\citeproctext]{ref-Hobfoll2002}
---------. 2002. {``Social and Psychological Resources and
Adaptation.''} \emph{Review of General Psychology} 6 (4): 307--24.
\url{https://doi.org/10.1037/1089-2680.6.4.307}.

\bibitem[\citeproctext]{ref-klaver2021exposure}
Klaver, M, BJ Van den Hoofdakker, H Wouters, G De Kuijper, PJ Hoekstra,
and A De Bildt. 2021. {``Exposure to Challenging Behaviours and Burnout
Symptoms Among Care Staff: The Role of Psychological Resources.''}
\emph{Journal of Intellectual Disability Research} 65 (2): 173--85.

\bibitem[\citeproctext]{ref-Lopez2024}
López, Leonardo, and Leonardo Giovanini. 2024. {``Adaptive Dynamic
Social Networks Using an Agent-Based Model to Study the Role of Social
Awareness in Infectious Disease Spread,''} July.
\url{https://doi.org/10.1101/2024.07.16.24310475}.

\bibitem[\citeproctext]{ref-mekonen2021estimating}
Mekonen, Tesfa, Gary CK Chan, Jason P Connor, Leanne Hides, and Janni
Leung. 2021. {``Estimating the Global Treatment Rates for Depression: A
Systematic Review and Meta-Analysis.''} \emph{Journal of Affective
Disorders} 295: 1234--42.

\bibitem[\citeproctext]{ref-Murase2021}
Murase, Yohsuke, Hang-Hyun Jo, János Török, János Kertész, and Kimmo
Kaski. 2021. {``Deep Learning Exploration of Agent-Based Social Network
Model Parameters.''} \emph{Frontiers in Big Data} 4 (September).
\url{https://doi.org/10.3389/fdata.2021.739081}.

\bibitem[\citeproctext]{ref-parsons2016cognitive}
Parsons, Sam, Anne-Wil Kruijt, and Elaine Fox. 2016. {``A Cognitive
Model of Psychological Resilience.''} \emph{Journal of Experimental
Psychopathology} 7 (3): 296--310.

\bibitem[\citeproctext]{ref-Romanyukha2023}
Romanyukha, Alexei Alexeevich, Konstantin Alexandrovich Novikov,
Konstantin Konstantinovich Avilov, Timofey Alexandrovich Nestik, and
Tatiana Evgenevna Sannikova. 2023. {``The Trade-Off Between COVID-19 and
Mental Diseases Burden During a Lockdown: Mathematical Modeling of
Control Measures.''} \emph{Infectious Disease Modelling} 8 (2): 403--14.
\url{https://doi.org/10.1016/j.idm.2023.04.003}.

\bibitem[\citeproctext]{ref-troy2023psychological}
Troy, Allison S, Emily C Willroth, Amanda J Shallcross, Nicole R
Giuliani, James J Gross, and Iris B Mauss. 2023. {``Psychological
Resilience: An Affect-Regulation Framework.''} \emph{Annual Review of
Psychology} 74 (1): 547--76.

\bibitem[\citeproctext]{ref-Yan2017}
Yan, Gang, Neo D. Martinez, and Yang-Yu Liu. 2017. {``Degree
Heterogeneity and Stability of Ecological Networks.''} \emph{Journal of
The Royal Society Interface} 14 (131): 20170189.
\url{https://doi.org/10.1098/rsif.2017.0189}.

\bibitem[\citeproctext]{ref-ZagerKocjan2021}
Zager Kocjan, Gaja, Tina Kavčič, and Andreja Avsec. 2021. {``Resilience
Matters: Explaining the Association Between Personality and
Psychological Functioning During the COVID-19 Pandemic.''}
\emph{International Journal of Clinical and Health Psychology} 21 (1):
100198. \url{https://doi.org/10.1016/j.ijchp.2020.08.002}.

\end{CSLReferences}




\end{document}
